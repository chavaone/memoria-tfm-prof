%

\chapter[Introdución]{Introdución}
Neste traballo describiremos o proceso de deseño dunha unidade didáctica para alumnos e a alumnas de 1º de ESO. Dita unidade didáctica está centrada na primeira parte do Bloque de Xeometría. Concretamente na definición, propiedades e relacións de elementos básicos de xeometría como puntos, rectas ou ángulos e tamén na clasificación e propiedades de polígonos prestando especial atención aos triángulos.

A \textbf{orixe} deste traballo está nas prácticas educativas que realizamos durante o mestrado. Nas prácticas tivemos a oportunidade de impartir clases de matemáticas a un grupo de alumnos e alumnas de 1º de ESO no IES. Leiras Pulpeiro de Lugo. Para impartir as clases realizamos un conxunto de actividades para que o alumnado adquirise os coñecementos necesarios de xeometría básica. Como froito da experiencia que supuxo o Practicum, estas actividades foron modificadas e fundamentadas coa teoría psicolóxica, pedagóxica e sociolóxica correspondente para logo seren plasmadas neste documento.

O traballo está dividido en catro capítulos que tratan diversas fases do deseño sendo esta introdución o primeiro deles.

No Capítulo~\ref{chap:fundamentacion} explicaremos a \textbf{fundamentación teórica} na que se basea o traballo presentado. Centrarémonos concretamente na fundamentación psicolóxica, coa que pretenderemos analizar o desenvolvemento psicolóxico das persoas a quen vai dirixido esta proposta e as diferentes teorías de aprendizaxe; na fundamentación pedagóxica da proposta, explicando en que teorías e estudos baseamos a nosa forma de ensinar, e, por último, na fundamentación sociolóxica, tratando o contexto sociocultural do alumnado.

No Capítulo~\ref{chap:desenvolvemento} explicaremos todas as \textbf{partes das que se compón a unidade didáctica} deseñada. Falaremos da fundamentación temática, isto é, por que se elexiu ese tema; e da xustificación dentro do currículo. Ademais detallaremos os distintos elementos formais dos que se compón a unidade: os seus contidos, forma de lograr as competencias clave e forma de avaliar. Falaremos da metodoloxía seguida, as medidas de atención a diversidade propostas para a unidade didáctica e tamén explicaremos cada unha das actividades propostas. Faremos ao final unha valoración da aplicación destas actividades durante o Prácticum.

No Capítulo~\ref{chap:valoracion} faremos unha \textbf{valoración sobre o desenvolvemento deste traballo e sobre o mestrado en xeral}. Realizaremos unha valoración e unhas conclusións sobre o deseño da unidade didáctica presentada neste Traballo Final de Mestrado, e tamén faremos valoracións xenéricas sobre o contraste entre o aprendido durante as aulas do mestrado e o experimentado nas aulas e o nivel competencial adquirido nos estudos previos para poder exercer o ensino das matemáticas.
