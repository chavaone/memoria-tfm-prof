\subsection{Xustificación Curricular}
%• Xustificación da unidade atendendo á lexislación vixente (decretos, reais decretos de ensinanzas mínimas, e ordes), a outros aspectos psicopedagóxicos e didácticos que xustificaren a súa inclusión no currículo da etapa, e/ou o curso e a súa contextualización.
Mediante a xustificación curricular pretendemos situar a presente unidade didáctica dentro da normativa legal existente así como establecer vínculos entre as orientacións designadas pola lexislación e as prácticas seguidas.

A \emph{Lei Orgánica 8/2013, do 9 de decembro, para a mellora da calidade educativa} (LOMCE a partir de agora) establece o marco lexislativo actual no que respecta a educación. Esta lei conta de un artigo único no que se establecen unha serie de modificacións que esta lei fai con respecto a anterior,\emph{Ley Orgánica 2/2006, de 3 de mayo, de Educación} (LOE a partir de agora). O \emph{Real Decreto 1105/2014, de 26 de decembro, polo que se establece o currículo básico da Educación Secundaria Obrigatoria e do Bacharelato}, como o seu nome indica estable un curriculo común a todo o estado español así como a distribución de comptencias en materia de educación entre goberno central e comunidade autónomas. En concreto, o goberno da Xunta de Galicia adapta este currículo xeral para todo o estado ao caso da comunidade galega no \emph{Decreto 86/2015, do 25 de xuño, polo que se establece o currículo da educación secundaria obrigatoria e do bacharelato na Comunidade Autónoma de Galicia}. A unidade didáctica deseñada está feita conforme a esta normativa legal.

Unha das \textbf{competencias clave} que se establece na LOMCE e tamén no Decreto 86/2015 é a ``Competencia matemática e competencias básicas en ciencia e tecnoloxía''. A materia de matemáticas ten un papel moi activo na consecución desta competencia. Segundo o Artigo 13 do Decreto 86/2015 que trata da organización do primeiro ciclo de educación secundaria obrigatoria establécese que Matemáticas é unha das materias troncais dos dous primeiros cursos da ESO. Tendo en conta que esta unidade está deseñada para alumnos e alumnos de 1º de ESO, Matemáticas será unha materia de oferta obrigatoria para este curso. Ademais do papel relevante desta competencia clave no ámbito das matemáticas traballaranse tamén as outras competencias marcadas pola lexislación como se pode ver na Sección~\ref{sec:competencias}.

O Decreto 86/2015 establece cinco bloques de \textbf{contidos} para 1º de ESO na materia de Matemáticas, ``Bloque 1. Procesos, métodos e actitudes en matemáticas'', ``Bloque 2. Números e álxebra'', ``Bloque 3. Xeometría'', ``Bloque 4. Funcións'' e ``Bloque 5. Estatística e probabilidade''. Durante esta unidade didáctica traballaremos contidos relativos ao Bloque 3 mais tamén de forma transversal contribuiremos ao logro de competencias correspondentes ao Bloque~1 e ao Bloque~2.

En canto aos contidos do Bloque 3, esta será a primeira unidade do curso relativa a este bloque polo que será introductoria trataranse contidos relativos aos elementos básicos do plano, os ángulos e os polígonos, así como outros elementos xeométricos como o círculo e a circunferencia. Na Sección~\ref{sec:contidos} tratarase en profundidade estos contidos.

Ademais de forma transversal ao tema traballaremos en reforzar conceptos que foron adquiridos en unidades didácticas previas como pode ser a resolución de problemas e os procesos de matematización de problemas da vida real, no referente ao Bloque 1 e as resolución de operación aritméticas e  as expresións en linguaxe alxebraica no que respecta ao Bloque 2.

A unidade didáctica presentada tamén comparte e intenta aplicar algún dos \textbf{principios metodolóxicos} que están plasmados no Artigo 11 do Decreto 86/2015. O citado decreto indica que ``a metodoloxía didáctica neste etapa será nomeadamente activa e participativa, favorecendo o traballo individual e o cooperativo do alumnado, así como o logro dos obxectivos e das competencias correspondentes''. A este respecto, a nosa unidade intenta que a clase non sexa un monólogo do profesor fomentando activamente a súa participación e realizando actividades onde se traballa en equipo é o alumnado é o protagonista do proceso de aprendizaxe. Ademais o citado decreto indica que nas programación didácticas se deberán fixar ``as estratexias que desenvolverá o profesorado para alcanzar os estándares de aprendizaxe avaliables previstos en cada materia e, de ser o caso, en cada ámbito, así como a adquisición das competencias''. Nesta unidade didáctica traballamos para intentar acadar os estándares de aprendizaxe que se poden ver na Sección~\ref{sec:estandares}. Outro dos principios metodolóxicos establece que ``a intervención educativa debe ter en conta como principio a diversidade do alumnado''. Na Sección~\ref{sec:diversidade} explicamos as medidas planificadas nesta unidade didáctica ante a diversidade presente na aula. Tamén se establece como principio metodolóxico a ``integración e uso das tecnoloxías da información e da comunidación na aula'', estas serán empregadas tanto polo profesorado como polo alumando como un imporatante recurso didáctico. Por último a lexislación fai referencia ao tratamento de certos temas transversais a todas as materias. Durante a Sección~\ref{sec:transversais} falaremos como traballamos na nosa unidade didáctica estos temas.

En canto ao proceso de \textbf{avaliación} que se detalla na Sección~\ref{sec:avaliacion}, o Decreto 86/2015 establece que esta debe ser ``continua, formativa e integradora''.  Esta unidade didáctica porá os medios para que a avaliación sexa un proceso máis da aprendizaxe do alumnado e disporá de actividades de reforzo que se realizarán no caso de detectar deficiencias na aprendizaxe dos nosos alumnso e alumnas.

Por último, subliñar que a unidade didáctica presentada neste traballo traballa activamente a prol da consecución dos \textbf{obxectivos da educación secundaria obrigatoria} establecidos no Artigo 10 do Decreto 86/2015. Neste decreto definense os obxectivos como ``referentes relativos aos logros que o alumnado debe alcanzar ao rematar o proceso educativo, como resultado das experiencias de ensino e aprendizaxe intencionalmente planificadas para tal fin''. A continuación veremos como pretendemos traballar algún dos seguintes obxectivos:

A través do traballo en equipos heteroxéneos no que se intentará mesturar alumnado de diferente sexo, etnia, nacionalidade e nivel de rendemento académico intentaremos traballar os seguintes obxectivos de etapa:

\begin{itemize}
    \item Asumir responsablemente os seus deberes, coñecer e exercer os seus dereitos no respecto ás demais persoas, practicar a tolerancia, a cooperación e a solidariedade entre as persoas e os grupos, exercitarse no diálogo, afianzando os dereitos humanos e a igualdade de trato e de oportunidades entre mulleres e homes, como valores comúns dunha sociedade plural, e prepararse para o exercicio da cidadanía democrática.

    \item Desenvolver e consolidar hábitos de disciplina, estudo e traballo individual e en equipo, como condición necesaria para unha realización eficaz das tarefas da aprendizaxe e como medio de desenvolvemento persoal.

    \item Valorar e respectar a diferenza de sexos e a igualdade de dereitos e oportunidades entre eles. Rexeitar a discriminación das persoas por razón de sexo ou por calquera outra condición ou circunstancia persoal ou social. Rexeitar os estereotipos que supoñan discriminación entre homes e mulleres, así como calquera manifestación de violencia contra a muller.

    \item Rexeitar a violencia, os prexuízos de calquera tipo e os comportamentos sexistas, e resolver pacificamente os conflitos durante os traballos en equipo.
\end{itemize}

A través da exposición do traballo feito en equipo pretendemos que o alumnado consiga o obxectivo:

\begin{itemize}
    \item Comprender e expresar con corrección, oralmente e por escrito, na lingua galega e na lingua castelá, textos e mensaxes complexas.
\end{itemize}

Realizando actividades nas que se lle pide á toda a clase que presenten como resolverían un determinado problema expoñendo as razóns polas que creen que a súa resposta é a correcta intentamos que o alumnado logre o obxectivo de etapa:
\begin{itemize}
    \item Desenvolver o espírito emprendedor e a confianza en si mesmo, a participación, o sentido crítico, a iniciativa persoal e a capacidade para aprender a aprender, planificar, tomar decisións e asumir responsabilidades.
\end{itemize}

Realizando actividades onde se lle pide aos alumnos e alumnas que procuren definicións en internet traballamos o obxectivo:
\begin{itemize}
    \item Desenvolver destrezas básicas na utilización das fontes de información, para adquirir novos coñecementos con sentido crítico. Adquirir unha preparación básica no campo das tecnoloxías, especialmente as da información e a comunicación.
\end{itemize}

Creando actividades nas que os alumnos se informan da historia de certo concepto matemático fomentamos un maior logro do obxectivo:

\begin{itemize}
    \item Coñecer, valorar e respectar os aspectos básicos da cultura e da historia propias e das outras persoas, así como o patrimonio artístico e cultural.
\end{itemize}

Empregando o galego como lingua vehicular na clase intentase traballar o obxectivo:

\begin{itemize}
    \item Coñecer e valorar a importancia do uso da lingua galega como elemento fundamental para o mantemento da identidade de Galicia.
\end{itemize}
