\subsection{Temporalización}
A temporalización da unidade didáctica indica ``cando se realiza e a súa duración''~\cite{delvalleud}. Nesta sección especificaremos que contidos do currículo de 1º de ESO teñen que ter adquiridos os nosos alumnos e alumnas para que poidamos poñer en práctica con éxito esta unidade didáctica, o período de tempo no que se desenvolveu durante as prácticas educativas do mestrado, e a duración da duración no tempo das actividades propostas por esta unidade didáctica.

Para a posta en práctica desta unidade didáctica é necesario que os e as estudantes teñan adquirido certas competencias con respecto á cálculo de expresións con números reais, raíces cadradas, potencias e linguaxe alxebraica.

Tendo en conta a planificación escolar feita polo centro onde aplicamos parte desta unidade, esta desenvolverase durante o mes de marzo e a primeira semana de abril tendo as vacacións de Semana Santa na metade da súa aplicación. Debido a isto as actividades da 0 a 6 formarán parte das cualificacións da segunda avaliación e o resto de actividades estarán dentro das cualificacións da terceira avaliación. Na Figura~\ref{fig:gantt:temp} pódese ver un diagrama de Gantt coa duración das actividades e a súa distribución no calendario.
\definecolor{foobarblue}{RGB}{0,153,255}

\begin{figure}[h!]
\begin{ganttchart}[
    vgrid={dashed,dashed,dotted,dotted,dotted,dotted,dotted},
    x unit=4.75mm,
    y unit title=5mm,
    y unit chart=6.5mm,
    canvas/.style={draw=none},
    title/.style={fill=gray, draw=none},
    title label font= \color{white}\scriptsize,
    title left shift=.1,
    title right shift=-.1,
    title top shift=.05,
    title height=.5,
    bar label font=\scriptsize,
    milestone label font=\scriptsize,
    group label font=\scriptsize\bfseries,
    bar/.append style={
        shape=rounded rectangle,
        inner sep=0pt,
        draw=foobarblue!50!black,
        very thick,
        top color=white,
        bottom color=foobarblue!50},
    bar left shift = 0.1,
    bar right shift = 0.1,
  ]{1}{34}
  \gantttitle{Marzo}{20} \gantttitle{Abril}{14} \\
  \gantttitlelist{5,...,19}{1}
  \gantttitlelist{27,...,31}{1}
  \gantttitlelist{1,...,14}{1} \\
  \ganttbar[inline, bar height=.6]{A1}{3}{3} \\
  \ganttbar[inline, bar height=.6]{A2}{3}{4} \\
  \ganttbar[inline, bar height=.6]{A3}{5}{7} \\
  \ganttbar[inline, bar height=.6]{A4}{10}{10} \\
  \ganttbar[inline, bar height=.6]{A5}{11}{12} \\
  \ganttbar[inline, bar height=.6]{A6}{13}{14} \\
  \ganttbar[inline, bar height=.6]{A7}{14}{14} \\
  \ganttbar[inline, bar/.append style={bottom color=red!50, draw=red!50!black}, bar height=.6]{S. Santa}{15}{17} \\
  \ganttbar[inline, bar height=.6]{A8}{18}{19} \\
  \ganttbar[inline, bar height=.6]{A9}{20}{21} \\
  \ganttbar[inline, bar height=.6]{A10}{24}{25} \\
  \ganttbar[inline, bar height=.6]{A11}{26}{26} \\
  \ganttbar[inline, bar height=.6]{A12}{27}{28} \\
  \ganttbar[inline, bar height=.6]{A13}{31}{32} \\
\end{ganttchart}
\caption{Temporalización das actividades da Unidade Didáctica}
\label{fig:gantt:temp}
\end{figure}

Vemos que esta unidade inclúe un total de 13 actividades con unha duración total de 21 sesións. A planificación amosada nesta figura é aproximada pois por unha parte dependendo da clase estas poden durar máis ou menos e por outra parte deixamos ao alumnado decidir as datas dos exames (Act. 5 e Act. 13).

