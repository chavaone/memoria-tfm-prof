% Responder como imos a ensinar
% Incluir fundamentación metodolóxica e curricular
% Fund. pedagógicos de tipo xeral
% Fund. pedagógicos concretos relacionadas coa temática.
% Tipos de actividades.
% Orientación metodolóxica.
% Justificación curricular (WTF!)

\section{Fundamentación Pedagóxica}

A pedagoxía é a ``disciplina que ten como obxecto de estudio a educación coa intención de organizala para cumprir con determindos fins, establecidados a partir dos que é desexable para unha sociedade, é dicir, o tipo de cidadan que se quere formar''\cite{wiki:pedagogia}. Nesta sección explicaremos os fundamentos pedagóxicos do presente traballo intentando desta forma responder a pregunta de como imos ensinar e porque o imos facer así. Falaremos de aspectos pedagóxicos xerais sobre que tipos de proceso ensino-aprendizaxe pensamos se deberían desenvolver na aula e explicaremos detalles relativos á aprendizaxe das matemáticas e da xeometría.

Na sección anterior falabamos de Ausubel como a persoa que formulou por primeira vez a teoría da aprendizaxe significativa. En Díaz Barriga~(1989 en \citeNP{arceo1998estrategias}) vemos que este autor entende que a aprendizaxe que se pode dar nunha aula pode definirse en dous eixes ou dimensións. Nun dos eixos contemplamos a forma en que se adquire a información. Esta pode ser por recepción ou por descubrimento. Se a aprendizaxe se produce por recepción producese cando os contidos se representan na súa forma final e se se produce por descubrimento, o contido principal no se da e é o alumnos ou alumna quen debe descubrilo. No outro eixo contemplamos a forma en que o coñecemento se incorpora na estrutura cognitiva do discente. Pode ser signigicativo, cando o novo coñecemento está directamente relacionado co que xa se posuía ou repetitivo cando se da por asocicións arbitrarias.

É común pensar que a aprendizaxe por recepción tende a ser repetitiva mais isto é un erro. Se se compren as condicións das que falabamos no apartado anterior (motivación e e que o coñecemento este organizado de forma significativa) podese xerar aprendizaxe significativo tanto por descubrimento como por recepción \cite{unedpsicoedu}. Durante a posta en práctica desta unidade didáctica intentaremos conseguir que a aprendizaxe sexa significativo e por recepción posto que este é o tipo de aprendizaxe que defende Ausubel que se debe dar nunha aula aínda que, como tamén explica o autor, tamén se poidan dar en certa proporción aprendizaxes por repetición e por descubrimento.
