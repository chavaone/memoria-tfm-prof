\subsection{Recursos, materiais e espazos}\label{sec:recursos}

En \citeA{materialesrecursosaula} vemos que un recurso entendese por ``calquer elemento, non deseñado especificamente para a aprendizaxe dun concepto ou procedemento determinado que o profesor ou profesora decide incorporar nas súas ensinanzas''. Por outro lado os mesmos autores sinalan que os materiais son os elementos que o profesorado emprega na aula e que son creados a tal fin. Por último os espazos son os lugares onde levamos a cabo a práctica educativa. Durante esta sección detallaremos os recursos, materiais e espazos que empregramos durante o desenvolvemento desta unidade didáctica.

Como lugar de aprendizaxe empregaremos unha aula normal do centro no que desenvolvemos esta unidade. Esta aula ten unha disposición tradicional dos pupitres e cadeiras, unha pequena biblioteca de aula que non empregaremos, encerado tradicional de xiz e un encerado dixital que está conectado ao ordenador do profesor. A aula tamén dispón dun armario onde se almacenan os ordenadores dos alumnos e alumnas.

Ademais, durante o desenvolemento da unidade empregaremos os seguintes recursos e materiais:

\begin{itemize}
    \item Fichas das actividades elaboradas polo profesorado e que se poden ver no Apéndice~\ref{chap:fich-act}.
    \item Blogue onde se publica información sobre as clases e cuxas entradas se poden ver no Apéndice~\ref{fich:blogue}.
    \item Programa de edición de imaxes instalado no ordenador do alumnado.
    \item O programa GeoGebra instalado no ordenador do profesor/a.
    \item Fotografías realizadas dos alumnos e do profesorado onde aparezan elementos xeométricos.
    \item Regla e compás tanto para os alumnos e alumnas como para que o profesor debuxe na pizarra.
    \item Triangulos manipulables construído con goma-eva.
    \item Fragmento do documental ``Pitágoras: Mucha más que un teorema''.
    \item Vídeo de YouTube con unha demostración con auga do Teorema de Pitágoras.
    \item Vídeos de YouTube no que se explica os mesmos contidos que os impartidos na clase para que os alumnos poidan repasar na casa.
\end{itemize}
