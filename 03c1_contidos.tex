\subsection{Contidos}\label{sec:contidos}
%separar contidos en bloques temáticos
%relacionalos coas competencias que queremos que se acaden
%ques sexan significativos para os alumnos
%incluir criterios de selección e ordenación

A secuencialización de \textbf{contidos} pretende responder a pregunta de que lle debemos ensinar aos alumnos. Intentaremos que durante o transcurso da implementación desta unidade didáctica, o alumnado adquira unha serie de conceptos, procedementos e actitudes. Durante esta proposta didáctica trataranse os seguintes contidos:

\begin{enumerate}[label=\bfseries Con\arabic*]
  \item\label{con1} Elementos básicos de xeometría. Punto, recta e plano.
  \item\label{con2} Posición relativa de rectas. Paralelismo e Perpendicularidade.
  \item\label{con3} Ángulos. Clasificación de ángulos en función da amplitude.
  \item\label{con4} Posición relativa de ángulos.
  \item\label{con5} Sistema sesaxesimal. Suma e resta de ángulos.
  \item\label{con6} Mediatriz e bisectriz.
  \item\label{con7} Polígono, concepto e partes. Clasificación polígonos por número de lados e ángulos.
  \item\label{con8} Triángulo. Clasificación triángulo por lados diferentes e ángulos.
  \item\label{con9} Suma dos lados dun triángulo.
  \item\label{con10} Puntos e rectas notables dun triángulo.
  \item\label{con11} Teorema de Pitágoras.
  \item\label{con12} Cuadrilátero. Clasificación cuadriláteros.
  \item\label{con13} Elementos dos polígonos regulares.
  \item\label{con14} Circunferencia, concepto e elementos.
  \item\label{con15} Círculo, concepto e elementos.
  \item\label{con16} Respecto polos compañeiros de traballo.
  \item\label{con16} Valoración da importancia da cooperación para realizar tarefas.
  \item\label{con17} Valoración da presenza das matemáticas en xeral e da xeometría en particular no día a día dos alumnos.
\end{enumerate}
