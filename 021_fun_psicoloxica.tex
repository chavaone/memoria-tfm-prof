% Responder a quen vai dirixido
% Aspectos relativos a la psicoloxía persoal del los alumnos dirixido a la idade de los alumnos del estudio.
% Desenvolvemento persoal/cognitivo de los estudantes
% Psicoloxía del aprendizaxe. Como se aprende.
% Autores como Piajet. Brousseau.


\section{Fundamentación Psicolóxica}

Segundo a \citeA{RAG}, psicoloxía é a ``ciencia que estuda o funcionamento psíquico a partir do comportamento do individuo así como os fenómenos de conduta e os procesos mentais para determinar as súas condicións e leis''. Posto que o propósito deste traballo de fin de mestrado é o deseño dunha unidade didáctica que pretende que un grupo de nenos e nenas adquiran unha serie de contidos, será de suma importancia coñecer as características da conduta e dos procesos mentais deste grupo de individuos. Nesta sección intentaremos analizar varios aspectos con respecto as peculiaridades psicolóxicas do noso alumnado.

A unidade deseñada está orientada a alumnos e alumnas de 1º de ESO. A idade da deles será maioritariamente de entre 11 e 12 anos salvo en determinados casos extraordinarios nos que pode ser inferior, por ter o alumno ou alumna altas capacidades e producirse unha aceleración; ou superior, debido a repetición dun curso ou a procedencia dun sistema educativo estranxeiro no que as competencias acadadas para a súa idade non son suficientes para avanzar de curso. En primeiro lugar faremos unha análise do concepto de adolescencia por estar o noso alumnado entrando neste período, a continuación falaremos do desenvolvemento cognitivo dos estudantes desta idade, do desenvolvemento da súa personalidade e, por último, explicaremos as teorías relativas a psicoloxía da aprendizaxe.

\subsection{Puberdade e adolescencia}
Como podemos ver en \citeA{adolescenciacoll} é importante diferenciar entre estes dous termos. Pois mentres a puberdade é un proceso biolóxico que comprenden un ``conxunto de cambios físicos que ao longo da segunda década de vida transforma o corpo infantil nun corpo adulto con capacidade para reproducirse''~(p.~436), a adolescencia é un ``período psicosociolóxico que se prolonga varios anos máis e que se caracteriza pola transición entra a infancia e a adultez''~(p.~436).

Estes autores sinalan que a puberdade é un fenómeno universal que atinxe a toda a especie humana e que causa alteracións físicas sobre todo nos caracteres sexuais primarios (órganos reprodutores) e nos secundarios (entre outros, cabelo facial, cambios na voz e ancheamento dos ombreiros nos homes e crecemento do peito e ancheamento das cadeiras nas mulleres). A adolescencia, por outro lado, afecta a unha certa porcentaxe da poboación mundial e o seu estudo estudo é relativamente recente. A adolescencia como un estadio particular do desenvolvemento non xorde ata finais do século XIX e principios do XX onde a industrialización fai que se lle de máis importancia á formación e os nenos e nenas comecen a pasar máis tempo nas escolas dependendo, durante este tempo, dos pais para a súa subsistencia. Na actualidade a adolescencia vense prolongado notablemente e cada vez máis tarde se adquire o status de adulto. Non obstante, noutras sociedades menos desenvolvidas que a nosa, existen ritos que duran varios días ou semanas nos que se pasa de neno a adulto. Neste caso non ten sentido falara de adolescencia no sentido que se lle da no mundo occidental.

No mesmo documento podemos ver que existen varias teorías sobre a adolescencia e mentres algúns autores e autoras fan fincapé en que son os cambios biolóxicos da puberdade os que provocan as transformacións psicolóxica que se viven na adolescencia, outros defenden que é o contexto social o que provoca estes cambios. Tampouco existe consenso nin ningunha teoría que explica que definitivamente o desenvolvemento nesta etapa. Porén, a percepción de que a esta etapa presenta un transo complicado para nenos e nenas está amplamente superada existindo varios estudos de que a maioría de seres humanos non presenta grandes dificultades neste período. En \citeA{adolescenciacoll} vemos que ``a porcentaxe de adolescentes que experimenta algún tipo de desaxuste psicolóxico non supera o 20\%''~(p.~440), sendo este porcentaxe similar ao de problemas na infancia.

\subsection{Desenvolvemento Cognitivo}
En \citeA{shaffer2000psicologia} vemos que na \emph{Teoría do Desenvolvemento Cognitivo} de Jean Piajet existen varias etapas do desenvolvemento cognitivo do ser humano, a etapa \emph{sesoriomotora} (ata os 2 anos), a etapa \emph{preoperatoria} (de 2 a 7 anos), a etapa das \emph{operacións concretas} (de 7 a 11 anos) e a etapa das \emph{operacións formais} (de 11 anos en adiante). Os nenos e nenas de 1º de ESO aos que vai dirixida este traballo teñen entre 11 e 12 anos polo que nos centraremos nas dúas últimas etapas dedicando máis atención á ultima.

Este libro explica que a etapa das operacións concretas caracterízase por que os nenos e nenas adquiren unha serie de operacións cognitivas que lles permitirán pensar en obxectos e acontecementos que experimentou no pasado. \citeA{piaget1997psicologia} cita como algunha das operacións que se adquiren durante este período: a conservación, a serialización, a clasificación, o número ou o espazo. Estas operacións non están de ningunha forma aisladas e son comúns a todos os individuos dun mesmo nivel mental. Non obstante este tipo de pensamento é limitado pois so se poden aplicar estes esquemas a obxectos reais ou imaxinables sendo incapaces de aplicalas a símbolos abstractos.

Estas limitacións que ten a etapa das operacións concretas son superadas na seguinte etapa, a das operacións formais. En \citeA{shaffer2000psicologia} explicase que esta etapa, que comeza a partir dos 11 anos, fai que os alumnos e alumnas aprendan operacións cognitivas formais e o pensamento deixará de estar vinculado ao observable podendo actuar sobre procesos e feitos hipotéticos. Estes autores sinalan que un operador formal pode traballar con hipóteses o cal é esencial para a aprendizaxe das matemáticas máis alá da aritmética simple. Nunha ecuación do tipo $2x + 5 = 15$, a variable $x$ non representa algo concreto polo que debe abordarse de forma abstracta.

Non obstante, existen varias críticas a esta teoría proposta por Piaget como podemos apreciar en \citeA{shaffer2000psicologia}. En primeiro lugar supón que todos os rapaces e rapazas chegan máis ou menos ao mesmo tempo e da mesma forma a esta etapa das operacións formais. O propio Piaget (1972 en \citeNP{shaffer2000psicologia}, p. 271) nos seus últimos anos de vida mencionou outra posibilidade, todos os estudantes chegan a este nivel pero so nos asuntos que son do seu interese ou que teñan unha importancia vital para eles. Esta hipótese vai en consonancia con estudos feitos posteriormente por outros autores como DeLisi e Staudt (1980 en \citeNP{cognitivocoll}, p.~463) que analizaron as diferencias do grado de alcance das operacións formais dependendo do tipo de preguntas e da formación dos entrevistados.

Por outro lado outra das críticas que recibe esta teoría é a case nula atención que fai as influencias sociais e culturais. O psicólogo ruso Lev Vygotsky formulou outra das teoría máis aceptadas e que si que ten en conta en gran medida as influencias sociais e culturais. Vemos en \citeA{shaffer2000psicologia} que segundo este autor, os nenos e nenas non desenvolven ``o mesmo tipo de mente en todo o mundo''~(p.~274), aprenden a empregar as súas capacidades para a resolución de problemas ``en conformidade coas normas e valores da súa cultura''~(p.~274). O psicólogo ruso cría que as habilidades máis importantes adquiridas polo alumnado proveñen da iteración con outras persoas do seu medio, sexan os seus proxenitores, profesores ou profesoras ou membros do seu grupo de iguais.

Este autor defende que nacemos con unhas funcións mentais elementais e que a cultura e a sociedade as transforma en funcións mentais superiores. Para facer isto cada cultura proporciona unhas ferramentas de adaptación intelectual que permiten mellorar as súas habilidades. No mesmo libro vemos que outro aspecto importante da teoría de Vygotsky é este concibía a aprendizaxe como cooperativo e identificada que os participantes máis expertos adaptaban o soporte que lle proporcionaban aos novatos en función da súa situación procurando sempre manterse na área de desenvolvemento proximal, que é ``a diferencia entre o que unha persoa pode lograr de forma independente e o que pode lograr cos consellos e estímulos dunha máis destra''~(p.~278).

\subsection{Desenvolvemento da Personalidade}
Os cambios físicos que os alumnos e alumnas viven na puberdade á súa idade causaran que a súa personalidade se vexa alterada sendo este período fundamental para a definición da súa personalidade futura como podemos ver en \citeA{personalidadcoll}. Algúns dos factores que se verán modificados durante esta época son o autoconcepto, a autoestima, a identidade persoal e o desenvolvemento moral.

Durante os primeiros anos da adolescencia, que é o período que viven os estudantes de 1º de ESO aos que lles impartimos clase, prodúcense unha serie de cambios físicos e psíquicos que terán repercusión sobre o seu autoconcepto. Durante estes anos os contidos deste autoconcepto soen estar relacionados con estes cambios que se producen polo que as referencias ao seu aspecto físico serán constantes. Tamén aparecerán frecuentemente as características ou habilidades sociais. En canto á estrutura do autoconcepto, vemos en \citeA{personalidadcoll} que nos primeiros anos da adolescencia xorden unhas ``primeiras abstraccións que integran características relacionadas''~(p.~473), estas abstraccións están compartimentadas e grazas a isto non detectarán as incompatibilidades delas evitando desta forma conflitos emocionais.

Relacionado co autoconcepto está a autoestima. Con respecto a isto, o mesmo autor sinala que durante a primeira etapa da adolescencia os niveis de autoestima descenderán de forma xeneralizada debido fundamentalmente a que o ou a adolescente non se sente satisfeito co seu corpo. O feito de que a sociedade sexa máis esixente co corpo da muller fai que nelas este descenso da autoestima sexa máis acusado. Outras razóns con respecto á menor autoestima está no cambio do colexio ó instituto co que conleva pasar de ser os alumnos de máis idade a ser os de menos, o cambio de profesores e de compañeiros e o aumento da competitividade non so académica senón na práctica de deportes. A todo este cóctel hai que engadirlle o inicio das relacións heterosexuais que engadirán mais presión e fará que se sintan máis inseguros.

Como podemos apreciar en Erikson (1968 en \citeNP{personalidadcoll}, p. 478) a construción da identidade persoal é a principal tarefa que deben resolver os adolescentes. Este concepto está relacionado co autoconcepto mais mentres o autoconcepto depende fundamentalmente do desenvolvemento cognitivo, a identidade é un fenómeno psicolóxico. Porén, a construción da identidade persoal durante as idades dos nosos alumnos e alumnas, que se atopan nos primeiros anos da adolescencia, ocupa un estado de difusión onde presentan pouco interese en cuestións relacionadas con ela.

O desenvolvemento moral do alumnado estará entre o que Kohlberg (1976 en \citeNP{personalidadcoll}) denomina nivel preconvencional, no que se se fai unha obediencia literal das normas debido a que pretenden evitar o castigo; e o nivel convencional, maís propio da adolescencia e no que se priman as expectativas do grupo social co obxectivo de conseguir a aprobación do mesmo. Por outro lado Gilligan (1982 en \citeNP{personalidadcoll}) defende un modelo diferenciado para as mulleres con niveis equiparables aos dos homes pero onde xorden outros intereses. A existencia de preocupacións morais nas mulleres diferentes dunha sociedade machista pode causarlles estres durante este período.

\subsection{Psicoloxía da aprendizaxe}
% centrarse en una teoría relevante para a proposta.
Ao longo do tempo o concepto de aprendizaxe foi definida de moi diferentes formas. Por exemplo, Good~e~Brophy~(1995 en \citeNP{unedpsicoedu},~p.~74) definen a aprendizaxe como ``un cambio relativamente permanente na capacidade de execución, adquirida por medio da experiencia''. Outra posible definición é a de Gagné (1976 en \citeNP{unedpsicoedu},~p.~74) que a define como ``o profeso que capacita aos organismos a modifica a súa conduta con unha certa rapidez nunha forma máis ou menos permanente''. Independentemente das múltiples definicións que se deron ao longo do tempo, existen uns elementos comúns a todas elas: a aprendizaxe consiste nun cambio, é froito da experiencia e este cambio é relativamente permanente \cite{unedpsicoedu}.

Pero, como se produce a aprendizaxe? Sobre esta cuestión téñense dado moitas teorías ao longo do tempo mais as máis relevantes foron as teorías condutuais e as teorías cognitivas. En \citeA{ertmer1993conductismo} vemos que para as \textbf{teorías condutistas} a aprendizaxe ``lógrase cando se demostra ou se exhibe unha resposta apropiada a continuación da presentación dun estímulo ambiental específico''~(p.~6). Para os condutistas ``toda actividade humana se explica en función de asociacións entre estímulos e respostas''~\cite[p.~82]{unedpsicoedu} o causa que os procesos superiores sexan reducidos a actividade muscular. Por outro lado, no \textbf{enfoque congnitivo}, ``a aprendizaxe equiparase a cambios discretos entre os estados de coñecemento máis que con cambios na probabilidade de resposta''~\cite[p.~9]{ertmer1993conductismo} como ocurria no enfoque condutista.

Mentres estas dúas teorías foron dominantes durante boa parte do século XX a finais deste século empeza a xurdir unha nova teoría que aporta unha visión diferente da forma en que aprendemos. O \textbf{construtivismo} equipara á aprendizaxe coa creación de significados a partir de experiencias (Bednar et al., 1991 en \citeNP{ertmer1993conductismo}). A pesar de que comparten coa vertente cognitivista que a aprendizaxe é unha actividade mental, consideran que a mente actúa como filtro da información que recibimos do mundo e que produce unha realidade propia (Jonassen, 1991a en \citeNP{ertmer1993conductismo}).

Unha das teorías construtivistas máis destacadas é a da \textbf{aprendizaxe significativa} formulada por Ausubel por primeira vez en 1962. Para este autor a aprendizaxe significativa prodúcese cando ``o contido de aprendizaxe se relaciona de modo non arbitrario, senón de forma substancial, cos coñecementos previos que (o alumnado) xa posúe''\cite[p.~206]{unedpsicoedu}. Para que se produza este tipo de aprendizaxe, requírense dúas condicións, por unha parte que haxa unha predisposición por parte do alumnado e por outra que a o material suxeito da aprendizaxe sexa lóxico e relacionado cos conceptos anteriormente aprendidos polos estudantes e, ademais, que existan conexións entre o material xa adquirido e o que se pretende adquirir \cite{rodriguez2004teoria}.

Ausubel (Ausubel, Novak e Hanesian, 1978 en \citeNP{unedpsicoedu}) entende que a aprendizaxe que se pode dar nunha aula pode definirse en dous eixos. Por un lado temos a aprendizaxe por recepción (os coñecementos presentase na súa forma final) fronte a aprendizaxe por descubrimento (o alumnado debe descubrir os coñecementos) e por outro lado está a aprendizaxe por repetición (asociacións de coñecemento arbitrarias) fronte á xa citada aprendizaxe significativa. Durante a posta en práctica desta unidade didáctica intentaremos conseguir que a aprendizaxe sexa significativo e por recepción posto que este é o tipo de aprendizaxe que defende Ausubel que se debe dar nunha aula aínda que, como tamén explica o autor, tamén se poidan dar en certa proporción aprendizaxes por repetición e por descubrimento \cite{unedpsicoedu}.
