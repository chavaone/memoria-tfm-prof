\subsection{Contribución ao logro das competencias clave}\label{sec:competencias}
As competencias son as características que adquire unha persoa para que sexa capaz de realizar distintas tarefas. En \citeA{aprsuperior} podemos ver que ``ademais de coñecementos e habilidades, a competencia implica a comprensión do que se fai e saber transferir''~(p.~3).

Na nosa proposta didáctica fomentamos a competencia de \textbf{Comunicación Lingüística (CCL)} obrigando ao alumnado a intervir na clase e a formular hipóteses sobre como pensa que se debería resolver un problema. Ao tratarse da materia de matemáticas, a \textbf{Competencia matemática e competencias básicas en ciencia e tecnoloxía (CMCCT)} trátanse durante toda a proposta. Ademais durante esta proposta empregamos frecuentemente o ordenador para diversas tarefas polo que a \textbf{Competencia Dixital (CD)} dos nosos alumnos é tratado. En menor medida intentamos que os alumnos adquiran competencias relativas a \textbf{Competencias sociais e cívicas (CSC)} e a \textbf{Aprender a aprender (CAA)} a través do traballo en grupo e da busca de información de algunha actividade.
