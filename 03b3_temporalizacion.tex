\subsection{Temporalización}
A temporalización consite en ... Para a posta en práctica desta unidade didáctica é necesario que os e as estudantes teñan adquirido competencias respecto ao uso de lenguaxe alxebraica. Tendo en conta a planificación escolar feita polo centro onde aplicamos parte desta unidade,, esta desenvolverase durante o mes de marzo e a primera semana de abril tendo as vacacións de Semana Santa na metade da súa aplicación. Debido a isto as actividades da 0 a 6 formarán parte das calificacións da segunda avaliación e o resto de actividades estarán dentro das calificacións da terceira avaliación. Na Figura~\ref{fig:gantt:temp} pódese ver un diagrama de Gantt coa duración das actividades e a súa distribución no calendario.

\begin{figure}[h!]
\centerline{
    \begin{ganttchart}[
        vgrid={dotted,dotted,dashed,dashed,dotted,dotted,dotted},
        x unit=4.5mm,
        y unit title=5mm,
        y unit chart=6mm,
        canvas/.style={draw=none},
        time slot format=isodate,
        title/.style={fill=gray, draw=none},
        title label font= \color{white}\scriptsize,
        title left shift=.1,
        title right shift=-.1,
        title top shift=.05,
        title height=.5,
        bar label font=\scriptsize,
        milestone label font=\scriptsize,
        group label font=\scriptsize\bfseries
      ]{2016-03-3}{2016-04-9}
      \gantttitlecalendar{month=name,day} \\
      \ganttbar{A0}{2016-03-4}{2016-03-6}  \\
      \ganttbar{A1}{2016-03-7}{2016-03-7} \\
      \ganttbar{A2}{2016-03-7}{2016-03-8} \\
      \ganttbar{A3}{2016-03-8}{2016-03-9} \\
      \ganttbar{A4}{2016-03-10}{2016-03-11} \\
      \ganttbar{A5}{2016-03-11}{2016-03-11} \\
      \ganttbar{A6}{2016-03-14}{2016-03-15} \\
      \ganttbar{A7}{2016-03-16}{2016-03-17} \\
      \ganttbar{A8}{2016-03-17}{2016-03-17} \\
      \ganttbar{A9}{2016-03-18}{2016-03-18} \\
      \ganttbar[inline, bar/.append style={fill=red!50}, bar height=.6]{S. Santa}{2016-03-19}{2016-03-28} \\
      \ganttbar{A10}{2016-03-29}{2016-03-29} \\
      \ganttbar{A11}{2016-03-30}{2016-03-31} \\
      \ganttbar{A12}{2016-03-31}{2016-04-1} \\
      \ganttbar{A13}{2016-04-4}{2016-04-5} \\
      \ganttbar{A14}{2016-04-6}{2016-04-7} \\
    \end{ganttchart}}
\caption{Temporalización das actividades da Unidade Didáctica}
\label{fig:gantt:temp}
\end{figure}

Vemos que esta unidade inclúe un total de 14 actividades con unha duración total de 18 sesións. A planificación amosada nesta figura é aproximada pois por unha parte dependendo da clase estas poden durar máis ou menos e por outra parte deixamos ao alumnado decidir as datas dos exames (Act. 6 e Act. 7).
