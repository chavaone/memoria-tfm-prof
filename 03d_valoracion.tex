%• Valoración da aplicación da unidade. Se por algunha circunstancia extrórdinaria non se puidese aplicar, a valoración referirase á idoneidade da unidade didáctica para o centro onde tiveron lugar as prácticas.
A maior parte das actividades desenvolvidas durante esta unidade didáctica foron postas en práctica durante o Practicum. Impartiuse en dous cursos de primeiro de ESO, un grupo grande de 18 estudantes e un agrupamento e so catro alumnos e alumnas. De seguido algunhas valoracións en canto a aplicación destas actividades.

Con respecto a \textbf{Actividade 0} (Sección~\ref{act0}), o resultado desta actividade foi peor do esperado. Por unha parte o alumnado expresou que non terminaba de entender o que se pedía polo que tivemos que explicar o que se pretendía lograr con esta actividade. Por outro lado, demostrouse que as alumnas e alumnos a pesar de teren case todos teléfono móbil e seren usuarios habituais de redes sociais como Instagram ou SnapChat, non tiñan un coñecemento sobre as tarefas ofimáticas máis básicas como pode ser a de enviar un correo electrónico. É subliñable neste sentido que un alumno confesou que este fora o primeiro correo electrónico que enviaba na súa vida. Ademais máis da metade dos alumnos non realizaron a actividade e non enviaron as fotografías. Por último, dentro das imaxes recibidas non había demasiada variedade e, por exemplo, para captar os triángulos, moitos alumnos recorrían a sinais de tráfico sendo isto un dos exemplos propostos durante a clase. Este feito provocou que fose necesario modificar algunha actividade na que era precisa unha certa variedade nas imaxes.

En canto aos resultados da \textbf{Actividade 2} (Sección~\ref{act2}), a parte realizada cos ordenadores foi de gran agrado para os alumnos segundo a enquisa de autoavaliación. Por outra parte isto pode ser debido simplemente a que tiñan que empregar os ordenadores non ao contido da actividade en si. Descubrimos que tamén había algún problema inicial por parte do alumnado para que empregasen o programa de edición proposto polo que sería interesante buscar unha alternativa que lles resultase máis sinxela. Ademais en canto ao tamaño dos equipos de traballo mencionar que os resultados foron moitos mellores no grupo reducido onde se traballou por parellas en lugar de grupos de tres a catro persoas. Esta última faceta foi resaltada pola propia titora que nos recomendou que traballasen por parellas.

Durante segunda parte da \textbf{Actividade 3} (Sección~\ref{act3}) xerouse unha certa dinámica competitiva entre un alumno e unha alumna do grupo grande se ben houbo outros alumnos que non participaron. En cambio no grupo reducido, o menor número de alumnos e alumnas permitiu que cada alumno clasificase varios ángulos diferentes e se practicasen máis estes conceptos.

As \textbf{Actividades 4 e 5} foron bastante aburridas e pouco interesantes para os alumnos polo que sería moi interesante buscar algunha alternativa que lles fose máis significativa.

En canto ao \textbf{exame} realizado dos conceptos básicos de xeometría, o exame realizado polo alumnado do grupo reducido foi diferente pois realizouse antes de explicar o concepto de mediatriz e bisectriz e os exercicios relacionados con este tema foron eliminados. Neste grupo tamén se detectou que parte do alumnado lle era moi complicado entender algún dos conceptos traballados cando non estaban dispostos por separado. A gráfica empregada no exame onde aparecen varias liñas paralelas, secantes e perpendiculares formando a súa vez ángulos diferentes, pareceulles moi difícil a pesar de tela utilizado con anterioridade nas explicacións.

Tivemos que modificar as \textbf{Actividades 7} (Sección~\ref{act7}) e \textbf{11} (Sección~\ref{act11}) debido a que nas fotos entregadas polo alumnado non había suficiente variedade. Como alternativa planteamos que en vez de buscar os polígonos nas fotos os debuxasen nunha folla en branco. Esta modificación causou que fose moito menos atractiva para o alumado. Unha mellor opción atopándose nesta situación sería procurar nós imaxes onde aparecesen máis variedade de polígonos para poder seguir empregando as imaxes.

Da \textbf{Actividade 9} (Sección~\ref{act9}) destacar que o alumnado participou activamente na busca de posibles solucións aos problemas plantexados e razoaron bastante adecuadamente os procedementos seguidos. Na \textbf{Actividade 10} (Sección~\ref{act10}) os alumnos quedaron moi impresionados con algunha das demostracións gráficas feitas nos vídeos proxectados.

En canto á \textbf{valoración global} feita polo alumnado a maioría valorou de forma positiva a nosa estadía no centro e destacaron como aspectos a mellorar, que explicase máis lento, que fixese máis exercicios e que mellorase a caligrafía. Por outro lado en conversacións informais cos estudantes, resaltaban que lles gustaría que as matemáticas fosen moito máis prácticas e comparábanas coa forma práctica coa que explicaba a bioloxía a súa profesora. A actividade que máis gustou a maioría de alumnos segundo a auto-avaliación foi o Trivial de Polígonos plantexado na Actividade 13 (Sección~\ref{act13}). Aínda que igual que pasou en actividades anteriores isto pode ser debido ao uso de ordenadores para realizala.

Como \textbf{reflexión global} sobre a posta en práctica desta proposta didáctica, dicir que quizais o fallo máis importante desta proposta didáctica sexa a pouca realización de exercicios similares aos do exame que se realizaron na aula. Sería moi interesante ver os resultados desta mesma proposta dedicando algunhas sesións a realización de exercicios. Neste sentido as críticas do alumnado son totalmente certas. Con respecto a velocidade de explicación e a caligrafía é dende logo un aspecto a corrixir na nosa práctica docente.

Con respecto as actividades propostas, pensamos que a maioría resultaron moi interesantes para o alumando e que lles acercaron a xeometría en particular e as matemáticas en xeral dunha forma diferente á que estaban acostumados.
