

\chapter{Valoración e Conclusións}\label{chap:valoracion}
%− Unha valoración dos resultados obtidos no desenvolvemento do TFM e unhas conclusións do mesmo.



%− Unha reflexión persoal que contiver, cando menos, ideas conclusivas a respecto dos seguintes aspectos: a) Contraste dos coñecementos conseguidos nas distintas materias do Mestrado e as experiencias das prácticas educativas

En canto ao \textbf{contraste existente entre o aprendido nas aulas do mestrado e o vivido durante a realización do Prácticum}, existen materias que foron realmente útiles durante a realización das prácticas, outras que non o foron tanto e algunhas onde os procedementos de ensino-aprendizaxe non fomos quen de conseguir trasladalos a aula real.

En primeiro lugar no módulo xenérico vimos nocións básicas de psicoloxía do desenvolvemento, de política educativa, de función titorial, do tratamento das linguas nas aulas e de didáctica. Estas materias foron de gran utilidade para introducirnos no mundo do ensino-aprendizaxe dende unha perspectiva que nunca vivíramos, a perspectiva do docente.

Desta forma adquirimos unha pequena introdución á psicoloxía do desenvolvemento e aos distintos trastornos que podemos atopar no noso día a día como docentes como poden ser o autismo ou outros trastornos do desenvolvemento, a superdotación, os trastornos de lectoescritura ou os trastornos da audición. Durante as clases de educación sociedade e política educativa vimos dende un punto de vista crítico a lexislación e a normativa que debe cumprir un docente durante o seu traballo así como os principios da sociedade nos que se asenta dita normativa. Nas clases de función titorial aprendimos o que debería facer un bo titor ou titora e diferentes técnicas para levar a cabo isto. Durante as leccións de política lingüística comprendemos a importancia que ten o trato que lle damos a unha lingua cando impartimos clases. Por último no módulo xenérico aprendemos uns conceptos xenéricos sobre didáctica e a organización dos centros escolares.

Todas estes aprendizaxes foron de extrema utilidade a hora de afrontar o Prácticum pois, como xa se comentou, conseguiron romper a barreira que supón pasar dende a perspectiva do discente a do docente. Durante o Prácticum e sobre todo na primeira parte, puidemos ver cos nosos propios ollos moitos dos aspectos tratados nestas materias. Pensamos é imprescindible resaltar a importancia destas materias dentro do mestrado pois supoñen unha base para coñecer tanto as institucións nas que imos traballar como para coñecer certas características dos nosos futuros alumnos e alumnas e como axudalos a que acaden o mellor rendemento posible.

Dentro da parte específica do mestrado, hai un conxunto de materias destinadas a ofrecer un complemento de formación sobre tecnoloxía tanto en ESO como en Bacharelato, un conxunto de materias pensadas para introducirnos na investigación e innovación docente, materias relacionadas coa didáctica da tecnoloxía e das matemáticas e co deseño de unidades didácticas e por último unha materia de innovación centrada no campo da tecnoloxía.

Durante as prácticas educativas moitas destas materias non as puxemos en práctica por diversas razóns, en primeiro lugar ao elixir a especialidade de matemáticas os complementos a formación de tecnoloxía tanto en ESO como en Bacharelato non foron empregados. Da mesma forma a formación en investigación e innovación docente non foi de utilidade durante as prácticas pois non se realizaron nelas proxectos importantes de investigación nin de innovación. Por outro lado as materias centradas na didáctica tanto de matemáticas como de tecnoloxía foron e gran utilidade pois aportaron gran cantidade de ideas para a realización de actividades durante a intervención autónoma da aula. Da mesma forma coa materia de proxectos de innovación adquirimos formación sobre certas metodoloxías ou ferramentas que podemos empregar para que o alumando adquira os coñecementos de forma máis atractiva para eles e elas.

Non obstante con respecto á perspectiva ofrecida durante algunhas aulas do mestrado, dende un punto de vista máis académico, dista bastante da realidade vivida no Prácticum. Polo menos coa formación que recibimos durante o mestrado e coa nula experiencia docente previa que tiñamos, parece canto menos complicado levar a cabo algunha das propostas didácticas de corte innovador que se propoñen en algunha das materias.

%TODO: valroación mestrado-realidade seguir...

%b) Reflexión sobre o nivel de desenvolvemento persoal das competencias adquiridas para ensinar dentro da especialidade docente.
Por outro lado en canto ao nivel de desenvolvemento persoal das \textbf{competencias para ensinar matemáticas}, consideramos que actualmente este nivel non é o optimo. A nosa formación é a dunha carreira técnica (Enxeñería Informática) con unha base matemática moi importante sobre todo nos campos da análise, a álxebra e a estatistica. Esta base debería capacitarnos para impartir clases de todos os bloques menos o de xeometría mais lamentablemente os coñecementos nestes campos foron impartidas nas materias dos primeiros anos da enxeñería e actualmente non nos lembramos da maior parte deles polo que para impartir clases destas partes sería necesario un amplo repaso que de seguro faremos durante a preparación da oposición.
