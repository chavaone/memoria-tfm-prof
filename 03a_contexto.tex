%Contextualizacón da UD

É importante coñecer o contexto no que está baseado esta unidade didáctica. Isto é a características do centro e a clase nas que se vai impartir.

Esta unidade didáctica está orientada para os alumnos dunha clase de \textbf{1º de ESO} do \textbf{IES.~Leiras~Pulpeiro} de Lugo. Trátase dun centro público de ensino secundario creado no ano 1998 nun dos barrios periféricos de Lugo. Na actualidade impártense nel clases de Ensino Secundario Obrigatorio, Bacharelato e Ciclo Formativo de Grao Superior de Industrias Alimentarias e conta con moi boa fama dentro dos institutos da cidade.

En canto as características socio-económicas da poboación do centro, como aparece no Proxecto Educativo de Centro, ``pola súa situación, unha zona periférica con poboación asentada en vivendas sociais, defínese unha estrutura socio-económica media-baixa''. O alumnado que acolle é variado, a súa poboación está formada por unha parte de nenos e nenas do barrio, con presenza da etnia xitana; outra parte da contorna rural, alumando inmigrante, con procedencia latinoamericana sobre todo; e en menor cantidade outro conxunto de alumnos que por motivos disciplinarios chega doutros centros.

En canto a estrutura física e a situación do instituto, este atópase situado nunha parcela de $8000 m^2$ dos cales está construídos uns $6600 m^2$. O espazo co que conta o instituto cobre ben as súas necesidades posibilitando incluso a formación de grupos reducidos para os alumnos con dificultades. As aulas están organizadas do xeito tradicional, se ben existen elementos na aula que pretenden certa innovación como a disposición de unha biblioteca de aula, un corcho onde os alumnos pegan os seus traballos e equipamento informático como ordenadores tanto para o profesor como para o alumnado (proxecto Abalar\footnote{\href{https://www.edu.xunta.es/espazoAbalar/espazo/proxecto-abalar/introducion}{edu.xunta.es/espazoAbalar/espazo/proxecto-abalar/introducion}}), proxector e encerado dixitais na maior parte de aulas.

%TODO: description alumnos.
