\subsection{Medidas de atención á diversidade}\label{sec:diversidade}
%• Atención á diversidade, coas estratexias e materiais para levala a cabo.

Víamos antes que un dos principios metodolóxicos formulados polo Decreto 86/2015 era que ``a intervención educativa debe ter en conta como principio a diversidade do alumnado''. Este é un valor clave para intentar acadar unha igualdade de oportunidades de todo o alumnado. Partindo disto consideramos unha serie de medidas para tomar coa intención de adaptar a cada alumno ou alumna que o alumnado os materiais atendendo as súas necesidades, capacidades e intereses; intentando deste xeito que desenvolvan ao máximo o seu potencial.

Como primeira medida para tomar será a formación dos grupos. Estes estarán formados de forma que neles se mesturen estudantes con diversas características logrando desta forma que uns se axuden aos outros equilibrando progresivamente os niveis e por outro lado fomentando a tolerancia.

No caso doutras transtornos da vista, procuraremos que estes alumnos ou alumnas se senten cerca do encerado e facilitarémoslles material cun tamaño de letra máis grande. Ademais no momento de usar o ordenador poderanse empregar as ferramentas de accesibilidade como o lector de pantalla do que dispón GNU/Linux.

En canto ao alumnado estranxeiro que teña problemas co idioma, facilitaráselles os materiais no seu idioma de orixe dentro do posible e no caso de que teñan problemas por un nivel académico máis baixo intentaremos proporcionarlles actividades complementarias adaptadas ao seu nivel.
