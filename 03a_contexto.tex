%Contextualizacón da UD

É importante coñecer o contexto no que está baseado esta unidade didáctica. Isto é, as características do centro e a clase nas que se vai impartir.

Esta unidade didáctica está orientada para os alumnos dunha clase de \textbf{1º de ESO} do \textbf{IES~Leiras~Pulpeiro} de Lugo. Trátase dun centro público de ensino secundario creado no ano 1998 nun dos barrios periféricos de Lugo. Na actualidade impártense nel clases de Ensino Secundario Obrigatorio, Bacharelato e Ciclo Formativo de Grao Superior de Industrias Alimentarias e conta con moi boa fama dentro dos institutos da cidade.

En canto as características socio-económicas da poboación do centro, como aparece no Proxecto Educativo de Centro, ``pola súa situación, unha zona periférica con poboación asentada en vivendas sociais, defínese unha estrutura socio-económica media-baixa''. O alumnado que acolle é variado, a súa poboación está formada por unha parte de nenos e nenas do barrio, con presenza da etnia xitana; outra parte da contorna rural, alumnado inmigrante, con procedencia latinoamericana sobre todo e, en menor cantidade, outro conxunto de alumnos que por motivos disciplinarios chega doutros centros.

 O espazo co que conta o instituto cobre ben as súas necesidades posibilitando incluso a formación de grupos reducidos para os alumnos con dificultades. As aulas están organizadas do xeito tradicional, se ben existen elementos na aula que pretenden certa innovación como a disposición de unha biblioteca de aula, un corcho onde os alumnos pegan os seus traballos e equipamento informático como ordenadores tanto para o profesor como para o alumnado (proxecto Abalar\footnote{\href{https://www.edu.xunta.es/espazoAbalar/espazo/proxecto-abalar/introducion}{edu.xunta.es/espazoAbalar/espazo/proxecto-abalar/introducion}}), proxector e encerado dixitais na maior parte de aulas.

A clase onde se desenvolverá esta unidade didáctica é un grupo de 23 estudantes de 1º de ESO. Do total de alumnos, a división por xénero é de 8 nenas e 15 nenos. Non hai ningún repetidor neste grupo polo que a idade dos alumnos será de entre 11 e 12 anos. Dos 23 alumnos e alumnas hai 4 que son de etnia xitana (dous nenos e dúas nenas) e que presentan certos problemas de absentismo e teñen mal comportamento en xeral, así como falta de motivación nos seus fogares. É importante prestar especial atención a estes estudantes para evitar unha posible e máis que probable situación de fracaso escolar. Por outro lado, tamén hai dentro da aula un alumno con autismo mais que a pesar do seu trastorno, os seus proxenitores decidiron que debía estar nunha aula ordinaria e non nunha aula especializada para alumnos con trastornos do desenvolvemento como a que existe no propio centro. Este alumno require especial atención pois é habitual que teña condutas disruptivas dentro da aula.

Non obstante, dentro do centro educativo fanse agrupamentos flexibles con alumnos con problemáticas especiais como as descritas anteriormente. En concreto, na nosa clase asisten a estes agrupamentos as dúas nenas de etnia xitana e un dos nenos de etnia xitana e o neno con autismo. Debido a isto, esta unidade didáctica non se aplicará a estes estudantes. Por outro lado, durante os meses anteriores ao inicio das prácticas educativas do mestrado os dous alumnos de etnia xitana foron expulsados de forma definitiva do centro por unha falta grave.

Tendo en conta todas estas circunstancias, o número total de alumnos cos que se farán as actividades desta unidade será de 18 onde 6 son rapazas e 12 rapaces. Este conxunto de alumnos presenta en xeral un bo comportamento e tenden a ser participativos dentro das clases.
