\subsection{Recursos}\label{sec:recursos}
Os recursos son os medios e material físico e lóxico (programas) necesario para que poidamos levar a cabo esta unidade didáctica. En concreto, empregaremos unha aula normal do noso centro na que hai un encerado de xiz e un encerado dixital conectado ao ordenador do profesor. Tamén precisamos do seguinte material:
\begin{itemize}
    \item Fichas das actividades que se poden ver no Apéndice~\ref{chap:fich-act}.
    \item Blogue onde se publica información sobre as clases.
    \item Ordenadores con acceso a internet e un programa de edición de imaxes instalado para o alumnado.
    \item Fotografías realizadas dos alumnos e do profesorado onde aparezan elementos xeométricos.
    \item Regla e compás tanto para os alumnos e alumnas como para que o profesor debuxe na pizarra.
    \item Triangulos manipulables construído con goma-eva.
    \item Fragmento do documental ``Pitágoras: Mucha más que un teorema''.
    \item Vídeo de YouTube con unha demostración con auga do Teorema de Pitágoras.
    \item Vídeos de YouTube no que se explica os mesmos contidos que os impartidos na clase para que os alumnos poidan repasar na casa.
\end{itemize}
