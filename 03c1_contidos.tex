\subsection{Contidos e temas transversais}\label{sec:contidos}
%separar contidos en bloques temáticos
%relacionalos coas competencias que queremos que se acaden
%ques sexan significativos para os alumnos
%incluir criterios de selección e ordenación

Nesta sección explicaremos os contidos que pretendemos que o noso alumnado adquira durante a posta en práctica desta unidade didáctica.

Os contidos son segundo \citeA{orientacionesgobvasco} un ``conxunto de procedementos, conceptos e actitudes que hai que desenvolver dunha forma integrada para alcanzar os obxectivos propios da área ou da materia''. Seguindo o mesmo criterio, \citeA{psicologiacurriculum} divide os tipos de contidos en conceptuais, procedementais e actitudinais. Este autor explica que os contidos conceptuais fan referencia a feitos, conceptos ou principios; os contidos procedementais a procedementos e os contidos actitudinais a valores, normas e actitudes.

Para \citeA{ensenharcomprender} os contidos conceptuais ``refírense ao conxunto de informacións que caracterizan a unha disciplina ou campo do saber''. A mesma autora indica que os contidos procedementais son ``o conxunto de accións ordenadas, orientadas á consecución dunha meta''. Por último, os contidos procedementais ``responden ao aspecto valorativo do coñecemento, ao compromiso persoal e social que implica o saber''.

Ademais dos contidos marcados para cada materia establecidos polo Decreto 86/2015, este mesmo decreto establece no seu Artigo 4 unha serie de temas transversais que ``se traballaran en todas as materias, sen prexuízo do seu tratamento específico nalgunhas das materias de cada etapa''. A continuación expoñeremos os contidos propios desta unidade didáctica ordenados de forma que lles resulten significativos aos alumnos e alumnas xunto coa forma en que traballamos os denominados elementos comúns do currículo:

Comezamos pedíndolles aos alumnos que expliquen que é para eles a xeometría e pedíndolles que procuren definicións en internet. Desta forma traballamos os seguintes contidos:

\begin{enumerate}[label=\bfseries Con\arabic*, align=left, leftmargin=1.5cm]
    \item\label{con:xeometria} Que é a xeometría?
    \item\label{con:buscadefinicions} Busca de definicións en internet.
    \item\label{con:participacionclase} Participación ordenada dentro da clase.
\end{enumerate}

Tratamos desta forma temas transversais como a \emph{expresión oral e escrita} debido a intervención do alumnado na clase ou ás \emph{tecnoloxías da información e da comunicación} a través da busca de información en internet.

Seguimos as sesións traballando a presenza de elementos básicos de xeometría como puntos, liñas e planos no mundo real tendo en conta as súas relacións. Para iso os alumnos sacarán fotos onde vexan certos elementos xeométricos para despois na clase traballar con estas fotos marcando estes elementos con un programa informático e expoñendo diante dos seus compañeiros os elementos destacados. Traballaremos desta forma os seguintes contidos:

\begin{enumerate}[label=\bfseries Con\arabic*, align=left, leftmargin=1.5cm]
    \setcounter{enumi}{3}
    \item\label{con:xeometriacontorna} Valoración da presenza da xeometría no día a día do alumnado.
    \item\label{con:elementosbasicos} Elementos básicos de xeometría. Punto, recta e plano.
    \item\label{con:semirecta} Concepto de semirrecta e de segmento.
    \item\label{con:posicionrectas} Posición relativa de rectas. Paralelismo e Perpendicularidade.
    \item\label{con:email} Uso do correo electrónico para enviar traballos.
    \item\label{con:gimp} Utilización dun programa de edición de imaxes para marcar elementos de imaxes.
    \item\label{con:companheiros} Respecto polos compañeiros de traballo.
    \item\label{con:cooperacion} Valoración da importancia da cooperación para realizar tarefas.
\end{enumerate}

Coas actividades descritas anteriormente tamén se traballan temas transversais como a \emph{expresión oral e escrita} a través da exposición do traballo realizado, as \emph{tecnoloxías da información e da comunicación} a través do uso do ordenador para marcar as imaxes e o feito de ter que compartilas a través do correo electrónico e a \emph{educación cívica constitucional}, \emph{igualdade de sexos} e \emph{resolución pacífica de conflitos} a través do fomento do traballo cooperativo e en grupo.

A continuación explicaremos o concepto de ángulo a partir do de recta e veremos o sistema sesaxesimal como forma de medir a amplitude dos ángulos. Por último veremos a clasificación dos ángulos en función da súa amplitude. Practicaremos este último aspecto a través dun pequeno concurso. Ademais do contido \ref{con:participacionclase}, traballaremos os seguintes:

\begin{enumerate}[label=\bfseries Con\arabic*, align=left, leftmargin=1.5cm]
  \setcounter{enumi}{11}
  \item\label{con:angulos} Ángulos. Clasificación de ángulos en función da amplitude.
  \item\label{con:posicionangulos} Posición relativa de ángulos.
  \item\label{con:sexasesimal} Sistema sesaxesimal. Suma e resta de ángulos.
\end{enumerate}

Neste caso como temas transversais tratados, traballaremos sobre todo a \emph{expresión oral e escrita} a través da participación no pequeno concurso que faremos para practicar a clasificación de ángulos.

Nas seguintes sesións estudaremos o concepto de mediatriz e a bisectriz como lugares xeométricos con unhas certas propiedades e aprenderemos a trazalos.

\begin{enumerate}[label=\bfseries Con\arabic*, align=left, leftmargin=1.5cm]
    \setcounter{enumi}{14}
    \item\label{con:mediatrizconp} Mediatriz e bisectriz. Concepto matemático.
    \item\label{con:mediatriztraz} Uso de regra e compás para trazar mediatrices e bisectrices.
\end{enumerate}

Continuamos o desenvolvemento da unidade didáctica vendo o concepto de polígono a partir do de segmento. Veremos a súa clasificación xeral para logo atender ao caso dos triángulos onde veremos a clasificación por número de lados e amplitude dos seus ángulos. Para traballar a clasificación dos triángulos o alumnado, de novo, deberá marcar no ordenador determinados elementos xeométricos. Debido a isto ademais dos contidos novos expostos a continuación volveranse a traballar os contidos \ref{con:gimp}, \ref{con:email}, \ref{con:companheiros} e \ref{con:cooperacion}:

\begin{enumerate}[label=\bfseries Con\arabic*, align=left, leftmargin=1.5cm]
  \setcounter{enumi}{16}
  \item\label{con:poligono} Polígono, concepto e partes.
  \item\label{con:clasificacionpol} Clasificación polígonos por número de lados e ángulos.
  \item\label{con:triangulo} Triángulo. Clasificación triángulo por lados diferentes e ángulos.
\end{enumerate}

Da mesma forma que antes ao empregar o ordenador, traballar en grupo e expoñer os resultados estaremos tratando os temas transversais relativos á \emph{expresión oral e escrita}, as \emph{tecnoloxías información e da comunicación}, a \emph{educación cívica constitucional}, a \emph{igualdade de sexos} e a \emph{resolución pacífica de conflitos}.

Nas sesións seguintes, mediante un debate onde se presentarán diversos problemas e os alumnos expoñerán oralmente as solucións traballaremos os puntos e rectas notables dun triángulo. Ademais do \ref{con:participacionclase} intentaremos que o alumnado adquira:

\begin{enumerate}[label=\bfseries Con\arabic*, align=left, leftmargin=1.5cm]
  \setcounter{enumi}{19}
  \item\label{con:puntosrectasnotables} Puntos e rectas notables dun triángulo.
\end{enumerate}

Mediante a actividade anterior fomentaremos a \emph{expresión oral e escrita} e a \emph{educación cívica constitucional} debido a que a alumnado deberá expoñer as súas solucións e explicalas.

Continuaremos analizando a historia do teorema de Pitágoras así como algunha das súas demostracións e utilidade resolvendo problemas da vida real:

\begin{enumerate}[label=\bfseries Con\arabic*, align=left, leftmargin=1.5cm]
  \setcounter{enumi}{20}
  \item\label{con:histpitag} Historia do Teorema de Pitágoras.
  \item\label{con:dempitag} Demostracións do Teorema de Pitágoras.
  \item\label{con:problepitag} Uso do Teorema de Pitágoras para a resolución de problemas.
\end{enumerate}

Unha vez rematado coa análise das propiedades dos triángulos, estudaremos a clasificación de cuadriláteros cunha actividade similar á empregada para a clasificación dos triángulos. Debido a isto ademais de repetir os contidos \ref{con:gimp}, \ref{con:email}, \ref{con:companheiros} e \ref{con:cooperacion}, traballaremos:

\begin{enumerate}[label=\bfseries Con\arabic*, align=left, leftmargin=1.5cm]
  \setcounter{enumi}{23}
  \item\label{con:cuadrilateros} Cuadrilátero. Clasificación cuadriláteros.
\end{enumerate}

Durante estas actividades volveremos a fomentar a \emph{expresión oral e escrita}, as \emph{tecnoloxías información e da comunicación}, a \emph{educación cívica constitucional}, a \emph{igualdade de sexos} e a \emph{resolución pacífica de conflitos} de xeito similar ao feito anteriormente.

Para rematar esta unidade didáctica veremos elementos xeométricos presentes en polígonos regulares así como o concepto de circunferencia e círculo e as súas propiedades:

\begin{enumerate}[label=\bfseries Con\arabic*, align=left, leftmargin=1.5cm]
  \setcounter{enumi}{24}
  \item\label{con:regulares} Elementos dos polígonos regulares.
  \item\label{con:circunferencia} Circunferencia, concepto e elementos.
  \item\label{con:circulo} Círculo, concepto e elementos.
\end{enumerate}

Ademais de todos os contidos tratados anteriormente, traballaremos o elemento transversal \emph{tecnoloxías información e da comunicación} mantendo un blogue onde se irán actualizando os contidos de cada sesión con información adicional.
