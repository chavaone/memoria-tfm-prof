% Respondera a para que imos ensinar.
% Contexto sociocultural
% Sociedad, democracia y educación, etc.
%cidaddanía crítica...

\section{Fundamentación Sociolóxica}

A educación é ante todo un fenómeno social. \citeA{guerraeducacion} defínena como ``o proceso de socialización a través do cal as persoas adquiren e desenvolven capacidades dirixidas con un fin social''~(p.~2). O Decreto 86/2015, do 25 de xuño, polo que se establece o currículo da educación secundaria na nosa comunidade establece como unha das finalidades da educación secundaria a de formar ao alumnado ``para o exercicio dos seus dereitos e das súas obrigas na vida como cidadáns e cidadás''. Tendo en conta isto, durante este apartado analizaremos os aspectos que, como docentes, debemos traballar nunha clase para que os cidadáns de mañá estean o mellor preparados posible para a súa vida adulta, que valores lles debemos inculcar e de que problemáticas da sociedade os e as debemos protexer.

En \citeA{gimeno2001hacia} definen a cidadanía que se pretende conseguir como unha que sexa ``inclusiva da diferenza e denunciadora de calquera tipo de exclusión''~(p.~24) e ``fundamentada na xustiza e na equidade''~(p.~24) facendo referencia a intención de eliminar calquera desigualdade presente na aula e conseguir que o noso alumando actúe noutros ámbitos na mesma dirección. Os autores falan tamén de que esta cidadanía sexa ``fortalecedora da identidade propia e aberta ao diálogo coas outras identidades e culturas''~(p.~24) polo que debe a par coñecer e respectar a súa cultura así como a dos demais. Outro dos aspectos dos que o autor fala é que esta cidadanía debe estar sostida polo ``compromiso e responsabilidade social na transformación da realidade''~(p.~25) de forma que debemos intentar formar alumnos e alumnas críticos e activos dentro da nosa sociedade.

A eliminación das desigualdades é posible se a educación actúa como un igualador social. Hai que ter en conta que vivimos nunha sociedade onde en moitos aspectos impera unha palpable desigualdade cunha vertente fundamentalmente económica. Esta desigualdade económica que sempre existiu, vemos en \citeA{elpaisclasebaja} que durante os últimos anos a raíz da crise económica aumentou de forma considerable. Sendo incluso alarmante a situación de moitos nenos e nenas que viven baixo o limiar da pobreza como se pode ver en \citeA{elpaispobreza}.

Ante esta desigualdade a educación debe intentar actuar, na medida do posible, como igualadora social. A propia LOMCE especifica no seu primeiro capítulo a equidade como un dos principios da educación en España. Indicando que debe garantir a ``igualdade de oportunidades para o pleno desenvolvemento da personalidade a través da educación'' ademais de actuar como un ``elemento compensador das desigualdades persoais, culturais, económicas e sociais''.

Aínda que a lei plasma este principio básico da educación, autores como \citeA{funcionessocieles} consideran que a igualdade de oportunidades ``non é outra cousa que un fermoso desexo ilusorio, unha mentira social, ou ben un discurso retórico da administración''~(p.~42). Este autor explica que ``a escola sempre fai unha dobre oferta, unha común e outra diferenciada''~(p.42), de forma que aínda que o currículo é común para toda a poboación estudantil, existen diferentes clases de escolas para diferentes niveis sociais. Estas diferenzas acentúanse coa maior ou menor capacidade que teñen os proxenitores para axudar aos seus fillos e fillas nas posibles dificultades que poidan xurdir na escola.

A pesar da visión negativa de \citeauthor{funcionessocieles}, consideramos que o noso papel como docentes debe ser minimizar no posible as posibles diferenzas entre o alumnado a través da ``posta en marcha de políticas compensatorias de acción discriminatoria positiva a favor de quen máis o necesita''~\cite[p.~17]{sacristan2000educacion}

En canto a coñecer a cultura propia e respectar a dos demais, debemos saber que unha das características da sociedade actual é que existe unha convivencia entre persoas de diferentes razas, etnias ou procedencias. Podemos ver que en \citeA{rivera2014practica} se resalta que ``os cambios políticos e económicos xeraron migracións dende distintas rexións no mundo e provocaron un sistema educativo culturalmente diverso''~(p.~72). O autor tamén resalta ``a necesidade de actuar e darlles a estas poboacións unha adecuada oferta educativa''~\cite[p.~72]{rivera2014practica}. Non obstante, neste contexto a oferta educativa adecuada é necesaria tanto para a poboación emigrante como para a local, pois mentres a poboación emigrante é probable que necesite unha maior atención por ter unha desigualdade económica cos seus compañeiros e compañeiras, a poboación local debe ser educada na tolerancia. Consideramos como se cita en \citeA{minguez1995valores}  que cómpre ``afrontar a educación para/e na tolerancia como núcleo básico dende ou co que se forma a cidadáns democráticos ou solidarios cos seus conxéneres, cultivando o respecto a mentalidades, culturas e persoas diferentes''~(p.~63).

Ademais de respectar e ser tolerantes coas culturas foráneas, debemos coñecer a propia. No caso da Comunidade Autónoma Galega, de igual forma que pasa naqueles territorios nos que haxa linguas minorizadas, pensamos que é fundamental un maior (en cantidade) e mellor (en calidade) uso da lingua propia de Galicia, o galego. O  Decreto 86/2015, do 25 de xuño, polo que se establece o currículo da educación secundaria obrigatoria establece como un dos obxectivos de etapa, ``comprender e expresar con corrección, oralmente e por escrito, na lingua galega e na lingua castelá, textos e mensaxes complexas, e iniciarse no coñecemento, na lectura e no estudo da literatura''. Este obxectivo implica que se pretende que o alumnado adquira unha igual competencia de uso das dúas linguas oficiais, e para que poidamos cumprilo con éxito, é preciso que adoptemos medidas. Autores como Bruner~(1983~en~\citeNP{vila2012algunhas}) resaltan que as linguas apréndense usándoas polo que estas medidas pasan por empregar as dúas linguas en todos os ámbitos posibles. Dada a situación de lingua minorizada en Galicia da lingua galega de forma que a súa presenza en moitos dos ámbitos é reducida, consideramos que a escola debe actuar neste punto como igualador empregándoa na maior cantidade de ámbitos posibles.

Por último, queremos fomentar un compromiso do noso alumnado coa sociedade á que pertence e formar desta forma cidadáns activos e críticos. A escola non deixa de ser un dos primeiros contactos dos nenos e nenas coa sociedade, polo que é interesante que dende o primeiro momento se habitúe aos futuros membros dela a ter condutas participativas e democráticas. Desta forma intentaremos converter a escola nunha sociedade embrionaria ou en miniatura \cite{diazrelaciones}. Mayordomo~(1998 en \citeNP{gimeno2001hacia}) sinala que en moitas experiencias existe a preocupación por desenvolver unha educación que consolide a democracia intentando erradicar da escola e da pedagoxía as posicións autoritarias. Fomentando actividades onde o alumnado expoña as súas solucións aos problemas fundamentando as razóns, estaremos traballando competencias importantísimas para os estudantes como futuros cidadáns.
