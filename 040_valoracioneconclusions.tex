

\chapter{Conclusións e reflexión persoal}\label{chap:valoracion}

Neste capítulo faremos unha valoración da experiencia que supuxo a elaboración deste traballo de fin de mestrado así como unhas conclusións del. Tamén faremos unha reflexión persoal en canto ao contraste existente entre o aprendido nas aulas do mestrado en contraste co que vimos e experimentamos durante a realización do practicum. Por último, faremos unha reflexión sobre o nivel de desenvolvemento das competencias necesarias para impartir clases na especialidade de matemáticas.

%− Unha valoración dos resultados obtidos no desenvolvemento do TFM e unhas conclusións do mesmo.
\section{Conclusións e valoración dos resultados}

Este TFM foi por un lado unha forma de plasmar nun documento parte dos coñecementos adquiridos nunha parte importante das materias do mestrado e por outra unha mellora e un afondamento da proposta didáctica levada a cabo nas prácticas académicas.

O mestrado reúne ensinanzas de temas moi variados: psicoloxía do desenvolvemento, política educativa, sociedade, organización escolar, didáctica en xeral e aplicada a temas concretos, acción titorial, tratamento das linguas, innovación na aula, etc. É un curso convulso onde se aprenden en poucos meses unha gran cantidade de conceptos novos sobre temas totalmente descoñecidos para a maioría no momento en que empezou o mestrado. Supón ademais, un choque importante para aqueles que veñen dunha carreira científico ou técnica e se atopan coa forma de traballar que hai nas carreiras de humanidades. A experiencia de ter que redactar este traballo supuxo un recorrido por todas as aprendizaxes levadas a cabo no mestrado para sermos quen de redactar un documento moi completo e que se apoia nunha sólida fundamentación teórica. Neste sentido, o traballo supón o colofón final do mestrado e unha mostra do que se aprendeu nel.

Por outro lado, a realización das prácticas académicas supuxo unha primeira inmersión no mundo educativo dende o punto de vista dos docentes. Durante ese mes e medio puidemos observar como funciona un instituto dende ese punto de vista e ademais levar a cabo unha proposta didáctica concreta. Como toda implementación dunha proposta didáctica, e máis cando é a primeira vez que se realiza, contén erros. Neste sentido, este TFM supón unha oportunidade para mellorar e afondar no traballo que realizamos nas prácticas curriculares corrixindo aqueles erros que atopamos e afondando tanto nas actividades que nos pareceron positivas como nunha fundamentación teórica necesaria.

Como valoración última sobre o que supuxo este traballo, dicir que pensamos que conseguimos mellorar notablemente a proposta que foi levada a cabo durante as prácticas a través dalgunhas modificacións que se poden ver na Sección~\ref{sec:aplicacion} e que conseguimos, ao mesmo tempo, redactar unha fundamentación psicolóxica, pedagóxica e sociolóxica que explique a nosa forma de ensinar facendo unha reflexión moi relevante sobre todo o aprendido durante o tempo que pasamos neste mestrado.

\section{Reflexión sobre o mestrado e o grao de adquisición das competencias profesionais}

Durante esta sección faremos unha pequena reflexión sobre o aprendido durante o mestrado e as diferencias entre o aprendido nas aulas e o vivido durante as prácticas académicas e tamén falaremos do nivel das competencias necesarias para impartir clase.

%− Unha reflexión persoal que contiver, cando menos, ideas conclusivas a respecto dos seguintes aspectos: a) Contraste dos coñecementos conseguidos nas distintas materias do Mestrado e as experiencias das prácticas educativas
En canto ao \textbf{contraste existente entre o aprendido nas aulas do mestrado e o vivido durante a realización do Prácticum}, existen materias que foron realmente útiles durante a realización das prácticas, outras que non o foron tanto e algunhas onde os procedementos de ensino-aprendizaxe non fomos quen de conseguir trasladalos a aula real.

En primeiro lugar no módulo xenérico vimos nocións básicas de psicoloxía do desenvolvemento, de política educativa, de función titorial, do tratamento das linguas nas aulas e de didáctica. Estas materias foron de grande utilidade para introducirnos no mundo do ensino-aprendizaxe dende unha perspectiva que nunca viviramos, a perspectiva do docente.

Desta forma adquirimos unha pequena introdución á psicoloxía do desenvolvemento e aos distintos trastornos que podemos atopar no noso día a día como docentes, como poden ser o autismo ou outros trastornos do desenvolvemento, a superdotación, os trastornos de lectoescritura ou os trastornos da audición. Durante as clases de educación, sociedade e política educativa vimos dende un punto de vista crítico a lexislación e a normativa que debe cumprir un docente durante o seu traballo así como os principios da sociedade nos que se asenta dita normativa. Nas clases de función titorial aprendemos o que debería facer un bo titor ou titora e diferentes técnicas para levar a cabo isto. Durante as leccións de política lingüística, comprendemos a importancia que ten o trato que lle damos a unha lingua cando impartimos clases. Por último no módulo xenérico aprendemos uns conceptos xenéricos sobre didáctica e a organización dos centros escolares.

Todas estas aprendizaxes foron de extrema utilidade á hora de afrontar as prácticas académicas pois, como xa se comentou, conseguiron romper a barreira que supón pasar dende a perspectiva do discente á do docente. Durante o Prácticum e sobre todo na primeira parte, puidemos ver cos nosos propios ollos moitos dos aspectos tratados nestas materias. Pensamos que é imprescindible resaltar a importancia destas materias dentro do mestrado pois supoñen unha base para coñecer tanto as institucións nas que imos traballar como para coñecer certas características dos nosos futuros alumnos e alumnas e como axudalos a que acaden o mellor rendemento posible.

Dentro da parte específica do mestrado, hai un conxunto de materias destinadas a ofrecer un complemento de formación sobre tecnoloxía tanto en ESO como en bacharelato e nas que estudamos de forma resumida os bloques de contidos que se imparten nas aulas de tecnoloxía durante estas dúas etapas; un conxunto de materias pensadas para introducirnos na investigación e innovación docente, nas que construímos de forma artificial proxectos de investigación e de innovación aprendendo de que elementos se compoñen estes proxectos; materias relacionadas coa didáctica da tecnoloxía e das matemáticas e co deseño de unidades didácticas, nas que vimos técnicas e estratexias para que o alumnado adquira mellor o coñecemento así como os pasos para a redacción de unidades didácticas e por último, unha materia de innovación centrada no campo da tecnoloxía, na que adquirimos coñecementos sobre unha serie de técnicas e modelos como poden ser a clase invertida ou as contornas persoais de aprendizaxe que poden ser útiles no noso día a día como docentes.

Durante as prácticas educativas, moitas destas materias non as puxemos en práctica por diversas razóns, en primeiro lugar ao elixir a especialidade de matemáticas os complementos á formación de tecnoloxía tanto en ESO como en bacharelato non foron empregados. Da mesma forma, a formación en investigación e innovación docente non foi de utilidade durante as prácticas, pois non se realizaron nelas proxectos importantes de investigación nin de innovación. Por outro lado, as materias centradas na didáctica tanto de matemáticas como de tecnoloxía foron de grande utilidade pois aportaron gran cantidade de ideas para a realización de actividades durante a intervención autónoma da aula. Da mesma forma, coa materia de proxectos de innovación adquirimos formación sobre certas metodoloxías ou ferramentas que puidemos empregar durante a nosa estadía no centro e que de seguro empregaremos con máis profundidade no futuro.

Con respecto á perspectiva ofrecida durante algunhas aulas do mestrado, quizais dende un punto de vista máis académico e ideal, dista bastante da realidade vivida no Prácticum. Polo menos coa formación que recibimos durante o mestrado e coa nula experiencia docente previa que tiñamos, parece canto menos complicado levar a cabo algunha das propostas didácticas de corte innovador que se propoñen nalgunha das materias. Hai que ter en conta que os que realizamos este mestrado na maioría levamos máis de vinte anos de experiencia na ensinanza como discentes e os métodos nos que nos ensinaron son puramente tradicionais sobre todo durante a carreira universitaria na que estivemos inmersos durante a última etapa das nosas vidas. Despois de tanto tempo, sen unha \emph{desintoxicación} deste modelo é complicado que desenvolvamos metodoloxías alternativas modernas tendo en conta, ademais, que as clases do mestrado duran pouco máis de catro meses e que as materias do mestrado onde teoricamente deberíamos aprender a desenvolver estas metodoloxías, as materias de didáctica sobre todo, teñen moi pouca carga lectiva.

Neste sentido consideramos despois de vivir a experiencia deste mestrado e de poñela en práctica nos institutos, que para a adquisición dunhas mellores competencias profesionais neste sentido o mestrado debería durar máis tempo ou de non ser así deberíase poñer énfase nas materias de didáctica quitándolle carga lectiva aos complementos a formación que son de pouca utilidade para o futuro desenvolvemento como profesor. Ademais sería de extrema utilidade que durante todas as aulas do mestrado se desenvolvese a metodoloxía que se pretende que nos como futuros profesores reproduzamos nas aulas, contribuíndo desta forma a unha certa desintoxicación.

%TODO: valroación mestrado-realidade seguir...

%b) Reflexión sobre o nivel de desenvolvemento persoal das competencias adquiridas para ensinar dentro da especialidade docente.
Por outro lado en canto ao nivel de desenvolvemento persoal das \textbf{competencias para ensinar matemáticas}, consideramos que actualmente este nivel non é o óptimo. A nosa formación é a dunha carreira técnica (Enxeñería Informática) cunha base matemática moi importante sobre todo nos campos da análise, a álxebra e a estatística. Esta base debería capacitarnos para impartir clases de todos os bloques menos o de xeometría, mais lamentablemente os coñecementos nestes campos foron impartidos nas materias dos primeiros anos da enxeñería e actualmente non nos lembramos da maior parte deles polo que para impartir clases destas partes sería necesario un amplo repaso que de seguro faremos durante a preparación da oposición.
