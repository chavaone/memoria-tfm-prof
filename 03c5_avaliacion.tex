\subsection{Avaliación}\label{sec:avaliacion}
%• Criterios e instrumentos de avaliación e seguimento da unidade.

Segundo \citeA{psicologiacurriculum}, a avaliación designa ``un conxunto de actuacións mediante as cales é posible axustar progresivamente a axuda pedagóxica ás características e necesidades dos alumnos e determinar se compriron ou non, e ata que punto, as intencións educativas que están na base da dita axuda pedagóxica''~(p.~125). Consideramos a partir desta definición que a avaliación non ten como misión poñer unha nota numérica, unha cualifiación, senón axudar a que o alumnado mellore as súas competencias, que aspectos debe mellorar e en cales o está a facer ben.

Durante esta sección explicaremos en que momentos fixemos a avaliación, os criterios que empregamos relacionándoos coas competencias clave que marca a lexislación, os estándares de aprendizaxe que intentamos acadar, as ferramentas empregadas para esta, os criterios de cualificación do alumnado, así como establecer instrumentos e parámetros para avaliar a execución da propia unidade didáctica.

A avaliación desta unidade didáctica lévase a cabo de forma inicial, continua e final. De forma \textbf{inicial}, intentaremos avaliar os coñecementos iniciais do alumnado na materia. Esta unidade didáctica pretende que os alumnos e alumnas adquiran coñecementos relativos á xeometría dende os seus aspectos máis básicos polo cal non será necesario ningún coñecemento de xeometría previo. Por outro lado, si será necesario certo dominio das operacións aritméticas con potencias, raíces cadradas e expresións alxebraicas, pero a avaliación destes conceptos xa estará feita durante a unidade didáctica propia desta temática que se debe impartir con anterioridade a esta. En canto á avaliación \textbf{continua}, intentaremos detectar posibles erros de compresión dos contidos que pretendemos que o alumnado adquira durante as actividades. De detectar estes erros, levariamos a cabo medidas correctoras. Ao \textbf{final} da unidade didáctica levaremos a cabo unha avaliación que terá por obxectivo medir o grao de adquisición das competencias desexadas para levar a cabo medidas correctoras como establecer a cualificación da unidade.

Os \textbf{criterios de avaliación} son ``as pautas que inciden na competencia do alumnado e permiten valorala de acordo cos retos co contexto actual''~\cite[p. 134]{secdidac}. En \citeA{orientacionesgobvasco} vemos que ``permiten observar con claridade o desenvolvemento das competencias básica dentro de cada materia''~(p.~26). O mesmo autor incide en que deben ser observables e estar relacionados coas competencias que pretendemos acadar. En concreto, deben estar relacionadas coas competencias clave marcadas pola LOMCE e o Decreto 86/2015 e que xa traballamos na Sección~\ref{sec:competencias}.

De seguido amosaremos os criterios de avaliación empregados relacionándoos coas citadas competencias clave entre parénteses:

\begin{enumerate}[label=\bfseries Cri\arabic*, align=left, leftmargin=1.5cm]
  \item\label{cri:definicions} Buscar definicións de termos matemáticos en internet con axilidade~(CMCCT, CAA e CD).
  \item\label{cri:interviraula} Participar de forma ordenada na clase respectando a quenda de fala dos compañeiros e falando con corrección.
  \item\label{cri:puntorecta} Identificar elementos xeométricos básicos como o punto a recta e o plano en exercicios escritos e na contorna~(CMCCT).
  \item\label{cri:resaltar} En grupos de traballo, resaltar elementos xeométricos empregando ferramentas informáticas adecuadas e compartir o resultado cos compañeiros~(CMCCT, CSC, CD e CCL).
  \item\label{cri:posrectas} Clasificar rectas en función da súa posición relativa~(CMCCT).
  \item\label{cri:angulos} Recoñecer a formación de ángulos na contorna~(CMCCT).
  \item\label{cri:sexasesimal} Empregar o sistema sesaxesimal para medir a amplitude de ángulos e calcular a suma e resta de ángulos~(CMCCT).
  \item\label{cri:clasangulos} Clasificar ángulos en función da súa amplitude~(CMCCT e CCL).
  \item\label{cri:mediatriz} Explicar as propiedades de mediatrices e bisectrices e saber trazar estes elementos~(CMCCT e CCL).
  \item\label{cri:poligonos} Identificar polígonos en exercicios escritos e na contorna. Diferenciar os seus elementos e clasificalos en función do número de lados e os seus ángulos~(CMCCT e CCL).
  \item\label{cri:claspoligonos} Clasificar triángulos e cuadriláteros en función das súas propiedades~(CMCCT e CCL).
  \item\label{cri:puntosnotables} Trazar os puntos e rectas notables dun triángulo e explicar as propiedades e a utilidade destes puntos e rectas~(CMCCT e CCL).
  \item\label{cri:pitagoras} Emprega o teorema de Pitágoras para a resolución de problemas xeométricos~(CMCCT).
  \item\label{cri:elementosregulares} Identificar elementos de polígonos regulares~(CMCCT e CCL).
  \item\label{cri:circunferencia} Explicar as propiedades das circunferencias e círculos e diferenciar os seus elementos~(CMCCT e CCL).
\end{enumerate}

En canto á \textbf{relación entre os criterios anteriores e as competencias clave}, como podemos ver case todos os criterios de avaliación anteriores miden en certa medida a adquisición da Competencia matemática e competencias básicas en ciencia e tecnoloxía (CMCCT) debido a que é o tema central que traballamos nesta unidade. Por outro lado, tamén se ten en conta bastante o nivel de adquisición da Competencia en Comunicación Lingüística (CCL) posto que se traballa bastante o vocabulario matemático e a expresión de argumentos de forma oral. Analizamos o nivel de adquisición da Competencia Dixital (CD) nos criterios que teñen relación co uso dos ordenadores e das Competencias Cívico Sociais (CCS) no momento en que se realiza un traballo en equipo. Por último a Competencia en Aprender a Aprender analizámola no criterio \ref{cri:definicions}, posto que se mide a capacidade do alumnos de realizar buscas en internet.

De todos os criterios de avaliación vistos anteriormente son considerados \textbf{mínimos esixibles} para a superación da unidade didáctica os criterios \ref{cri:puntorecta}, \ref{cri:posrectas}, \ref{cri:sexasesimal}, \ref{cri:clasangulos}, \ref{cri:mediatriz} \ref{cri:poligonos}, \ref{cri:claspoligonos}, \ref{cri:puntosnotables}, \ref{cri:pitagoras}, \ref{cri:elementosregulares} e \ref{cri:circunferencia}.

No Decreto 86/2015 defínense os \textbf{estándares de aprendizaxe} como ``especificacións dos criterios de avaliación que permiten definir os resultados de aprendizaxe e que concretan o que o alumnado debe saber, comprender e saber facer en cada disciplina''. Os estándares de aprendizaxe para esta unidade didáctica son os seguintes:

\begin{enumerate}[label=\bfseries Est\arabic*, align=left, leftmargin=1.5cm]
    \item\label{est:definicions} Busca definicións de termos matemáticos en internet con axilidade.
    \item\label{est:turnopalabra} Respecta a quenda de palabra no momento de intervir na aula.
    \item\label{est:falacorrecion} Fala con corrección e nos termos matemáticos adecuados.
    \item\label{est:puntorecta} Identifica elementos xeométricos básicos como o punto a recta e o plano en exercicios escritos e na contorna.
    \item\label{est:resaltar} En grupos de traballo, resalta elementos xeométricos en fotografía empregando ferramentas informáticas adecuadas.
    \item\label{est:compartir} Comparte o traballo realizado no ordenador cos compañeiros empregando os recursos informáticos adecuados.
    \item\label{est:posrectas} Clasifica rectas en función da súa posición relativa.
    \item\label{est:angulos} Recoñece ángulos en exercicios escritos e na contorna.
    \item\label{est:sexasesimal} Emprega o sistema sesaxesimal para medir a amplitude de ángulos.
    \item\label{est:sumarestaangulos} Calcular a suma e resta de ángulos empregando o sistema sesaxesimal.
    \item\label{est:clasangulos} Clasifica ángulos en función da súa amplitude.
    \item\label{est:mediatriz} Identifica mediatrices e bisectrices en figuras e sabe trazar elementos.
    \item\label{est:poligonos} Identifica polígonos en exercicios escritos e na contorna.
    \item\label{est:elementospoligonos} Identifica elementos dos polígonos como os vértices, lados e diagonais.
    \item\label{est:claspoligolados} Clasifica polígonos en función do número de lados e os seus ángulos.
    \item\label{est:claspoligonos} Clasifica triángulos e cuadriláteros en función das súas propiedades.
    \item\label{est:puntosnotables} Traza os puntos e rectas notables dun triángulo e explica as propiedades e a utilidade destes puntos e rectas.
    \item\label{est:pitagoras} Empregar o teorema de Pitágoras para a resolución de problemas xeométricos.
    \item\label{est:elementosregulares} Identifica elementos de polígonos regulares.
    \item\label{est:circunferencia} Explica as propiedades das circunferencias e círculos e diferencia os seus elementos.
  \end{enumerate}

Na Táboa~\ref{tab:ele} podemos ver unha relación entre os contidos que se detallaron na Sección~\ref{sec:contidos} e os criterios de avaliación e os estándares de aprendizaxe avaliables explicados con anterioridade.

\begin{table}[h!]
    \begin{tabular}{ l | l | l }
        \large{\textbf{Contidos}} & \large{\textbf{Crit. de Avaliación}} & \large{\textbf{Est. de Aprendizaxe}} \\ \hline
        \ref{con:xeometria} e \ref{con:buscadefinicions}             & \ref{cri:definicions}        &  \ref{est:definicions}\\
        \ref{con:participacionclase}                                 & \ref{cri:interviraula}       &  \ref{est:turnopalabra} e \ref{est:falacorrecion} \\
        \ref{con:xeometriacontorna}, \ref{con:elementosbasicos}
        e \ref{con:semirecta}                                        & \ref{cri:puntorecta}         &  \ref{est:puntorecta} \\
        \ref{con:posicionrectas}                                     & \ref{cri:posrectas}          & \ref{est:posrectas} \\
        \ref{con:email}, \ref{con:gimp}, \ref{con:companheiros}
        e \ref{con:cooperacion}                                      & \ref{cri:resaltar}           & \ref{est:resaltar} e \ref{est:compartir}\\
        \ref{con:angulos}                                            & \ref{cri:angulos}            & \ref{est:angulos} \\
        \ref{con:posicionangulos}                                    & \ref{cri:clasangulos}        & \ref{est:clasangulos} \\
        \ref{con:sexasesimal}                                        & \ref{cri:sexasesimal}        & \ref{est:sexasesimal} e \ref{est:sumarestaangulos} \\
        \ref{con:mediatrizconp} e \ref{con:mediatriztraz}            & \ref{cri:mediatriz}          & \ref{est:mediatriz} \\
        \ref{con:poligono} e \ref{con:clasificacionpol}              & \ref{cri:poligonos}          & \ref{est:elementospoligonos} e \ref{est:claspoligolados} \\
        \ref{con:triangulo} e \ref{con:cuadrilateros}                & \ref{cri:claspoligonos}      & \ref{est:claspoligonos} \\
        \ref{con:puntosrectasnotables}                               & \ref{cri:puntosnotables}     & \ref{est:puntosnotables} \\
        \ref{con:histpitag}, \ref{con:dempitag}
        e \ref{con:problepitag}                                      & \ref{cri:pitagoras}          &  \ref{est:pitagoras}\\
        \ref{con:regulares}                                          & \ref{cri:elementosregulares} &  \ref{est:elementosregulares}\\
        \ref{con:circunferencia} e \ref{con:circulo}                 & \ref{cri:circunferencia}     &  \ref{est:circunferencia}\\
    \end{tabular}
    \centering
    \caption{Relación entre Contidos, Criterios de Avaliación e Estandares de Aprendizaxe.}
    \label{tab:ele}
\end{table}

Para medir o grao de complección destes estándares empregamos unha serie de \textbf{ferramentas de avaliación}. As ferramentas ou instrumentos de avaliación son ``os medios que se empregan no proceso de ensino aprendizaxe para recoller información significativa''~\cite[p.~29]{orientacionesgobvasco}. Estas ferramentas deben ser o suficientemente variadas como para responder a diversidade do alumnado presente na aula. Para a avaliación desta unidade didáctica empregaremos:

\begin{itemize}
    \item \textbf{Proba escrita individual.} O alumnado deberá completar dúas probas escritas de forma individual onde deberá identificar os elementos xeométricos traballados na clase, clasificalos en función das súas propiedades, así como realizar exercicios prácticos relacionados coa materia. As probas escritas realizadas pódense ver nos Apéndices \ref{fich:ex1a} e \ref{fich:ex2}.
    \item \textbf{Proba por parellas empregrando TIC.} Os alumnos e alumnas deberán facer un pequeno trívial feito coa plataforma online Socrative. Por parellas e cada parella cun ordenador irán completando todas as preguntas do trívial. Cada vez que se complete unha pregunta, a plataforma amosará a resposta correcta de forma que o alumnado recibe un \emph{feedback} moi rápido. Empregaranse varios tipos de preguntas: preguntas onde necesitarán marcar unha resposta correcta entre varias opcións, outras onde terán que dicir se un enunciado é correcto ou non ou algunhas nas que deberán formular a súa propia resposta.
    \item \textbf{Resolución e explicación de exercicios no encerado.} Despois da realización individual dos problemas e exercicios propostos o alumnado sairá ao encerado a explicarlles aos seus compañeiros e compañeiras como se realiza un exercicio de forma correcta.
    \item \textbf{Exposición oral de traballos en grupo.} Despois de certas actividades onde os alumnos e alumnas traballarán en grupos de tres ou catro persoas identificando elementos xeométricos en fotos do entorno, deberán explicar os resultados do seu traballo aos demais compañeiros e compañeiras.
    \item \textbf{Observación.} A través da observación directa comprobaremos o traballo que cada alumno e alumna fai dentro do grupo de traballo así como detectar e corrixir erros que poida haber.
\end{itemize}

En canto á \textbf{cualificación} do alumnado, segundo \citeA{orientacionesgobvasco}, a cualificación é ``unha das decisións que se derivan do proceso da avaliación. É a expresión codificada que conforma a escala de valoración establecida pola normativa''~(p.~31). Os criterios de cualificación marcan o peso que cada un dos criterios de avaliación marcados anteriormente ten na nota final. Nesta unidade didáctica, os criterios de avaliación non esenciais non terán peso na nota final. Na seguinte lista podemos ver cal é a porcentaxe da nota final que ven dada por cada un dos criterios de avaliación:

\begin{itemize}
    \item \ref{cri:puntorecta}         (Identificar elementos xeométricos básicos como o punto \dots) $\,\to\,$ 0\%
    \item \ref{cri:posrectas}          (Clasificar rectas en función da súa posición relativa \dots) $\,\to\,$ 10\%
    \item \ref{cri:sexasesimal}        (Empregar o sistema sesaxesimal para medir a amplitude \dots) $\,\to\,$ 15\%
    \item \ref{cri:clasangulos}        (Clasificar ángulos en función da súa amplitude \dots) $\,\to\,$ 15\%
    \item \ref{cri:mediatriz}          (Explicar as propiedades de mediatrices e bisectrices e saber \dots) $\,\to\,$ 10\%
    \item \ref{cri:poligonos}          (Identificar polígonos en exercicios escritos e na contorna \dots) $\,\to\,$ 0\%
    \item \ref{cri:claspoligonos}      (Clasificar triángulos e cuadriláteros en función das súas  \dots) $\,\to\,$ 20\%
    \item \ref{cri:puntosnotables}     (Trazar os puntos e rectas notables dun triángulo e explicar  \dots) $\,\to\,$ 5\%
    \item \ref{cri:pitagoras}          (Emprega o teorema de Pitágoras para a resolución de   \dots) $\,\to\,$ 5\%
    \item \ref{cri:elementosregulares} (Identificar elementos de polígonos regulares  \dots) $\,\to\,$ 15\%
    \item \ref{cri:circunferencia}     (Explicar as propiedades das circunferencias e  \dots) $\,\to\,$ 5\%
\end{itemize}

En canto a \textbf{avaliación da execución da unidade didáctica}, empregamos dous métodos ou ferramentas, en primeiro lugar consideramos unha proba do bo ou malo funcionamento desta unidade a propia avaliación que lles fagamos aos nosos alumnos e alumnas pois nos permitirá ver se logramos os obxectivos, é dicir, que eles adquiriron os contidos que buscabamos. Por outra parte tamén nos parece relevante a súa opinión polo que ao finalizar a unidade didáctica pedirémoslle ao alumnado que responda un cuestionario coas seguintes preguntas:

\begin{itemize}
    \item Cales foron  as actividades que máis che gustaron?
    \item Cales foron as actividades que menos che gustaron.
    \item Que aspectos mellorarías das actividades que fixemos?
    \item Que nota (do 0 ao 10) lle poñerías a esta unidade didáctica?
\end{itemize}

Desta forma saberemos que actividades, segundo o alumnado, debemos mellorar e en que aspectos debemos facelo. Tamén aprenderemos as actividades que resultan interesantes para o alumnado e poderemos intentar crear actividades similares no futuro. Coa nota da unidade didáctica obteremos un valor da satisfacción xeral do alumando coa nosa práctica docente.
