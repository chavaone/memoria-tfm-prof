\subsection{Fundamentación temática}
%• Fundamentación mediante a achega relevante e actualizada de documentos que versaren sobre a temática elixida e sustentada en achegas de investigación didáctica.

Nesta sección explicaremos a importancia do tema elexido para a elaboración da unidade didáctica (a xeometría) no desenvolvemento dos nosos alumnos e alumnas.

Xa no século XVII, un dos filósofos, astrónomos e físicos máis destacados da historia, Galileo Galilei no seu libro \emph{Il Saggiatore}~(Galilei, 1623 en \citeNP{cursogeometria}) subliñaba que:
\begin{quote}
\vspace{-0.3\baselineskip}
A filosofía está escrita neste vasto libro que sempre está aberto ante os nosos ollos: refírome ao universo; pero non pode ser lido ata que non aprendamos a linguaxe e nos familiaricemos coas letras en que está escrito. Está escrito en linguaxe matemático, e as letras son triángulos, círculos e outras figuras xeométricas, sen as cales é humanamente imposible entender unha soa palabra
\vspace{-0.6\baselineskip}
\end{quote}

Esta importancia que xa é coñecida dende fai moito tempo, aparece marcada nos Principios e Estándares para a Educación Matemática da NCTM (2000 en \citeNP{cursogeometria}) onde se destaca que a esta da xeometría ``ofrece medios para describir, analizar e comprender o mundo e ver a beleza nas súas estruturas''.

Esta rama das matemáticas é útil para desenvolver o entendemento dos espazos, coñecer as formas e as posicións dos obxectos do mundo; mellorar a a habilidades respecto aos números e medidas pois o coñecemento da xeometría é indispensable para facer apreciacións é calculos sobre medidas e posicións de obxectos, mellorar a habilidade visual e no lugar de traballo pois moitas áreas como poden ser a arquitectura, o deseño ou a topografía fan un uso intensivo da xeometría. Ademais tamén a xeometría mellora as nosas capacidade de pensamente en tres dimensións así como empregar asiduamente todas as zonas do cerebro por ter a xeoemtría unha compoñente artística e, como gran parte das matemáticas, unha compoñente lóxica \cite{shockingreasons}.

Dada a importancia que ten a xeometría no desenvolvemento dos e das estudantes, a presente unidade didáctica contribuirá notablemente a este desenvolvemento pois con ela pretendemos que os alumnos adquiran os conceptos básicos que rixen o funcionamento da xeometría en dúas dimensións.
