\subsection{Avaliación}\label{sec:avaliacion}
%• Criterios e instrumentos de avaliación e seguimento da unidade.

Os \textbf{criterios de avaliación} son ``as pautas que inciden na competencia do alumnado e permiten valorala de acordo cos retos co contexto actual''~\cite[p. 134]{secdidac}. Hai que resaltar que consideramos que a avaliación ten como misión non so poñer unha nota numérica senón sobre todo axudar a que o alumnado mellore as súas competencias indicándolle en que actividades obtivo un bo rendemento e en cales se debe incidir máis. Os criterios empregados nesta proposta son:

\begin{enumerate}[label=\bfseries Cri\arabic*]
  \item\label{cri1} Recoñecer elementos básicos de xeometría como punto, recta, ángulo. Clasificar estes elementos atendendo as súas propiedades e a súa posición relativa.
  \item\label{cri2} Calcular a suma e resta de ángulos expresados en unidades de sistema sesaxesimal.
  \item\label{cri3} Explicar as propiedades de mediatrices e bisectrices e saber trazar estes elementos.
  \item\label{cri4} Diferenciar elementos dos polígonos e clasificalos en función do número de lados e os seus ángulos.
  \item\label{cri5} Clasificar triángulos e cuadriláteros en función das súas propiedades.
  \item\label{cri6} Trazar os puntos e rectas notables dun triángulo e explicar as propiedades e a utilidade destes puntos.
  \item\label{cri7} Empregar o teorema de Pitágoras para a resolución de problemas xeométricos.
  \item\label{cri8} Explicar as propiedades das circunferencias e círculos e diferenciar os seus elementos.
  \item\label{cri9} Calcular a área e o perímetro de figuras planas a través da descomposición en polígonos.
\end{enumerate}

\paragraph{Estándares de aprendizaxe}
%Obxectivos e competencias básicas.
Da mesma forma, en \cite{dogcurrlomce} defínense os estándares de aprendizaxe como ``especificacións dos criterios de avaliación que permiten definir os resultados de aprendizaxe e que concretan o que o alumnado debe saber, comprender e saber facer en cada disciplina.''.  Os estándares de aprendizaxe para esta unidade didáctica son os seguintes:

\begin{enumerate}[label=\bfseries Est\arabic*]
 \item\label{est1} Recoñece e describe elementos de xeometría tales como o punto, a recta ou o ángulo.
 \item\label{est2} Clasifica rectas e ángulos en función da súa posición relativa.
 \item\label{est3} Clasifica ángulos en función da súa amplitude.
 \item\label{est4} Emprega o sistema sesaxesimal para o cálculo de sumas e restas de ángulos.
 \item\label{est5} Define as características de mediatriz e bisectriz e sabe trazalas.
 \item\label{est6} Identifica elementos dos polígonos.
 \item\label{est7} Clasifica polígonos en función do número de lados e os seus ángulos.
 \item\label{est8} Clasifica triángulos e cuadriláteros en función das súas propiedades.
 \item\label{est9} Define os puntos e rectas notables dun triángulo e sabe trazalas.
 \item\label{est10} Resolve problemas de xeometría empregando o teorema de Pitágoras.
 \item\label{est11} Explica as propiedades dos puntos dunha circunferencia.
 \item\label{est12} Identifica elementos dunha circunferencia e dun círculo tales como arco, corda ou sector.
\end{enumerate}
