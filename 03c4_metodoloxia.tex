\subsection{Metodoloxía}

A metodoloxía é o ``conxunto de decisións acerca da organización do proceso de ensino aprendizaxe''~\cite[p.~18]{orientacionesgobvasco}. Engloba aspectos como o tipo de actividades, o agrupamento do alumnado, os recursos a empregar, os procedementos de avaliación ou o papel do profesor e do alumno~\cite{orientacionesgobvasco}. Para a realización desta unidade didáctica empregaranse varias estratexias ou métodos de ensino-aprendizaxe que, seguindo o visto durante a fundamentación pedagóxica (Sección~\ref{sec:pedago}), intentan, dentro do posible, que os alumnos e alumnas adquiran os coñecementos de forma significativa.

En paralelo ás actividades levadas a cabo na clase, construíuse e actualizouse un blogue cos contidos dados na clase. Neste blogue ademais de expoñer de forma sintética unha explicación sobre o dado cada día, inclúense vídeos de profesores explicando estes contidos. Este modelo está moi relacionado coa \textbf{aula invertida} ou \textbf{flipped classroom}. En \citeA{saez2014experiencia} vemos que este modelo consiste en ``prover aos alumnos de materiais audiovisuais, que lles resulten atractivos, e que lles faciliten os coñecementos teóricos que na ensinanza tradicional o profesor lles ofrecía na clase''~(p.~1). Na metodoloxía de aula invertida, o tempo de clase pasa a estar reservado para resolver dúbidas, incidir máis nos contidos máis difíciles para o alumnado ou reforzar o aprendido a través do material audiovisual aportado. Consideramos que esta metodoloxía levada a cabo na súa totalidade non é acertada no noso contexto pois por unha parte o seu éxito depende da responsabilidade dos alumnos e por outra parte aumentaría en gran medida o número de horas que o alumando ten que traballar na casa. Non obstante si que consideramos útiles usar vídeos para que os alumnos poidan repasar na casa algún concepto que non lles quedou claro de forma moito máis amena.

Nalgunhas das actividades empregamos a \textbf{gamificación} para motivar aos alumnos. En \citeA{diaz2013potencial} vemos como a gamificación ``a través do uso de certos elementos presentes nos xogos que os xogadores incrementen o seu tempo nel así como a súa predisposición psicolóxica a seguir nel''. As características do alumnado actual na súa condición de residentes dixitais \cite{residentesdigitales} fai que necesiten maiores doses de motivación e predisposición para a aprendizaxe. Neste sentido, vemos en \citeA{gamificacion2} que ``a conxugación adecuada destes elementos (gratificación, recoñecemento social, relación social, etc.) coa necesidade de motivación parece apuntar de forma case inexorable a dar unha importancia significativa á introdución do xogo na aprendizaxe''. Durante esta proposta didáctica a gamificación emprégase en maior ou menor medida durante as Actividades~3~(Sección~\ref{act:angulos}) e na Actividade~12~(Sección~\ref{act:trivial}). Nas dúas actividades detectouse como o nivel de implicación e de interese do alumnado aumentaba considerablemente con respecto ao resto de actividades propostas.

A xeometría é un campo da matemática moi adecuado para a utilización de \textbf{materiais e recursos manipulables} para explicala. Segundo \citeA{moreiro2010materiales},  o material manipulativo facilita os procesos de ensino-aprendizaxe dos alumnos, pois os alumnos experimentan situacións de aprendizaxe de forma manipulativa, que lles permite coñecer, comprender e interiorizar as nocións estudadas, por medio de sensacións. Durante a serie de actividades propostas empregouse material manipulativo durante a Actividade 7~(Sección~\ref{act:sumangulos}).

Outro dos recursos que se pode empregar na matemática e que pode colaborar en gran medida en sacala dentro do marco teórico no que parece estar metida é a \textbf{fotografía}. En \citeA{gonzalez1989fotografia} descríbese unha experiencia levada a cabo neste sentido tamén no campo da xeometría. O autor resalta a importancia deste tipo de actividades nas que o fundamental non é a matemática, pero que poñen ao alumnado en contacto con ela conseguindo sacala da aula e facerlle ver que existe na vida real.

Por último, fomentaremos dentro do posible o traballo en grupo de forma colaborativa e cando este non sexa posible e teñamos que aplicar unha metodoloxía máis tradicional de profesor transmisor de coñecemento intentaremos que o alumnado participa o máis posible volvéndose a clase, máis que nun monólogo do profesor, nunha conversa entre este e o alumnado.
