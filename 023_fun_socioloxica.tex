% Respondera a para que imos ensinar.
% Contexto sociocultural
% Sociedad, democracia y educación, etc.
%cidaddanía crítica...

\section{Fundamentación Sociolóxica}

A tarefa máis importante de calquera profesor ou profesora de secundaria e sen dúbida a de formar cidadáns. O Decreto 86/2015, do 25 de xuño, polo que se establece o currículo da educación secundaria na nosa comunidade establece como unha das finalidades da educación secundaria a de formar ao alumnado ``para o exercicio dos seus dereitos e das súas obrigas na vida como cidadáns e cidadás''. Mais non simplemente cidadáns no sentido xurídico e legal, pois que neste sentido todos e todas o somos, senón fomentando a tolerancia, e equidade entre persoas, o compromiso social ou a sustentabilidade. \citeA{gimeno2001hacia} definen a cidadanía que se debe formar como:

\begin{quote}
    \vspace{-0.5\baselineskip}
    Unha cidadanía inclusiva da diferencia e denunciadora de calquera tipo de exclusión. Fundamentada na xustiza e na equidade, mais sen esquecer a liberade. Fortalecedora da identidade propia e aberta ao diálogo coas outras identidades e culturas. [...] Sostida polo compromiso e a responsabilidade social na transformación da realidade e xestadora dun poder compartido e exercido dende a lóxica do servizo e non dende a concentración e a centralización.~(p.~24-25)
    \vspace{-0.6\baselineskip}
\end{quote}

Desta forma é preciso que traballemos intentando conseguir que a educación actúe como igualador social, fomentando a tolerancia entre os nosos alumnos e alumnas e formando rapaces e rapazas críticos e activos dentro da sociedade.

En canto á visión da educación como igualador social, hai que ter en conta que vivimos nunha sociedade onde en moitos aspectos impera unha palpable desigualdade cunha vertente fundamentalmente económica. Esta desigualdade económica que sempre existiu, vemos en \citeA{elpaisclasebaja} que durante os últimos anos a raíz da crise económica aumentou de forma considerable. Sendo incluso alarmante a situación de moitos nenos e nenas que viven baixo o limiar da pobreza como se pode ver en \citeA{elpaispobreza}.

Ante esta desigualdade a educación debe intentar actuar, na medida do posible, como igualadora social. A propia LOMCE especifica no seu primeiro capítulo a equidade como un dos principios da educación en España. Indicando que debe garantir a ``igualdade de oportunidades para o pleno desenvolvemento da personalidade a través da educación'' ademais de actuar como un ``elemento compensador das desigualdades persoais, culturais, económicas e sociais''.

Aínda que a lei plasma este principio básico da educación autores como \citeA{funcionessocieles} consideran que a igualdade de oportunidades ``non é outra cousa que un fermoso desexo ilusorio, unha mentira social, ou ben un discurso retórico da administración''~(p.~42). Este autor explica que ``a escola sempre fai unha dobre oferta unha común e outra diferenciada''~(p.42) de forma que aínda que o curriculum é común para toda a poboación estudantil, existen diferentes clases de escolas para diferentes niveis sociais. Estas diferencias acentúanse coa maior ou menor capacidade que teñen os proxenitores para axudar aos seus fillos e fillas nas posibles dificultades que poidan xurdir na escola.

A pesar da visión negativa de \citeauthor{funcionessocieles}, consideramos que o noso papel como docentes debe ser minimizar no posible as posibles diferencias entre o alumnado a través da ``posta en marcha de políticas compensatorias de acción discriminatoria positiva a favor de quen máis o necesitan''~\cite[p.~17]{sacristan2000educacion}

Outra das características da sociedade actual é que existe unha convivencia entre persoas de diferentes razas, etnias ou procedencias como podemos ver en \citeA{rivera2014practica} onde se resalta que ``os cambios políticos e económicos xeraron migracións dende distintas rexións no mundo e provocaron un sistema educativo culturalmente diverso''~(p.~72). O autor tamén resalta ``a necesidade de actuar e darlles a estas poboacións unha adecuada oferta educativa''~\cite[p.~72]{rivera2014practica}. Non obstante, neste contexto a oferta educativa adecuada é necesaria tanto para a poboación emigrante como para a local pois mentres a poboación emigrante é probable que necesite unha maior atención por ter unha desigualdade económica cos seus compañeiros e compañeiras, a poboación local debe ser educada na tolerancia. Para \citeA{minguez1995valores} cómpre ``afrontar a educación para/e na tolerancia como núcleo básico dende ou co que se forma a cidadáns democráticos ou solidarios cos seus conxéneres, cultivando o respecto a mentalidades, culturas e persoas diferentes''~(p.~63).

Consideramos, ademais, indispensable a formación de cidadáns activos e críticos. A escola non deixa de ser un dos primeiros contactos dos nenos e nenas coa sociedade polo que é interesante que dende o primeiro momento se habitúe aos cidadáns a ter condutas participativas e democráticas intentando converter a escola nunha sociedade embrionaria ou en miniatura \cite{diazrelaciones}. Mayordomo (1998 en \citeNP{gimeno2001hacia}) sinala que en moitas experiencias existe a preocupación por desenvolver unha educación que consolide a democracia intentando erradicar da escola e da pedagoxía as posicións autoritarias. Consideramos que é interesante sempre dentro do posible é necesario fomentar un espírito participativo dentro das clases que o alumando ante os problemas que poidan xurdir expoña as súas solucións e que as fundamente. Este tipo de actividades fomentarán competencias importantísimas para os estudantes como futuros cidadáns.

Para rematar, resaltamos un tema que afecta especialmente a comunidade autónoma galega mais tamén a todos aqueles territorios con linguas minorizadas, pensamos que é fundamental para un mellor desenvolvemento dos nosos alumnos e alumnas un maior (en cantidade) e mellor (en calidade) uso da lingua galega nas aulas de secundaria. O Decreto 86/2015, do 25 de xuño, polo que se establece o currículo da educación secundaria obrigatoria establece como un dos obxectivos de etapa, ``comprender e expresar con corrección, oralmente e por escrito, na lingua galega e na lingua castelá, textos e mensaxes complexas, e iniciarse no coñecemento, na lectura e no estudo da literatura''.  Para que poidamos cumprir con éxito este obxectivo, é preciso que adoptemos medidas para que o alumando adquira a mellor competencia posible nas dúas linguas. Autores como Bruner~(1983~en~\citeNP{vila2012algunhas}) resaltan que as linguas apréndense usándoas polo que  as medidas a aplicar pasan por empregar a lingua que está minorizada e, polo tanto, nunha situación de desigualdade na maior cantidade de ámbitos posibles.
