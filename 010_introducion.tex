% 

\chapter[Introdución]{Introdución}
Neste traballo describiremos o proceso de deseño dunha unidade didáctica para alumnos e a alumnas de 1º de ESO. Dita unidade didáctica está centrada na primeira parte do Bloque de Xeometría. Concretamente na definición, propiedades e relacións de elementos básicos de xeometría como puntos, rectas ou ángulos e taméń na clasificación e propiedades de polígonos prestando especial atenciaón aos triángulos.

Este traballo está dividido en catro capítulos que tratan diversas fases do deseño sendo esta introdución o primeiro deles.

No Capítulo~\ref{chap:fundamentacion} trataremos a fundamentación teórica na que se basea o traballo presentado. Centraremonos concretamente na fundamentación psicolóxica coa que pretenderemos analizar o desenvolvemento psicolóxico das persoas a quen vai dirixido esta proposta. Tamén explicaremos a fundamentación pedagóxica da proposta explicando en que teorías e estudos baseamos a nosa forma de explicar. Por último tamén trataremos a fundamentación sociolóxica centrandonos no contexto sociocultural do alumnado.

No Capítulo~\ref{chap:desenvolvemento} explicaremos as partes das qeu se compón a unidade didáctica proposta.

No Capítulo~\ref{valoracion} ...