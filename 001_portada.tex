%
% Portada.
%

% Nota: Sería más cómodo emplear el comando \maketitle que genera una portada de forma automática, pero
% no incluye toda la información que es necesario incluir en la memoria de un proyecto de fin de carrera
% de la Facultad de Informática de A Coruña.
%

\begin{titlepage}

	\begin{center}

		% Logotipo de la universidad.
		\includegraphics[width=6cm]{./eps/logo_udc.eps}
                \vspace{0.45cm}

		% Nombre de la facultad, de la universidad y del departamento en que se realiza el PFC.
		{\Large{\textbf{Universidade da Coruña}}}
		\\
		{\it \large{\textbf{Mestrado en Profesorado de Educación Secundaria Obrigatoria e Bacharelato, Formación Profesional e Ensino de Idiomas}}}
		\vspace{2.5cm}

                {\Huge Unidade Didáctica: ``Figuras xeométricas planas. Concepto e propiedades''} \\ \vspace{0.5cm}
                {\large Unidad Didáctica: ``Figuras geométricas planas. Concepto y propiedades''} \\ \vspace{0.15cm}
                {\large Didactic Unit: ``Plane geometric figures. Concept and properties''}


		% Indicamos el nombre de la titulación oficial que hemos cursado con tanto esfuerzo.
			\end{center}
                        \begin{bottompar}
                          \begin{flushright}
                            \begin{tabular}{rl}
                              \large{\textbf{Alumno:}}	&
			      \large{Chavarría Teijeiro, Marcos} \\

                              &
                              \small{(DNI: 33558386-Y)} \\

                              \large{\textbf{Especialidade:}}	&
                              \large{Tecnoloxía} \\

			\large{\textbf{Titora (Universidade):}}	&
			\large{Pérez González, Mercedes} \\

			\large{\textbf{Centro:}}	&
			\large{IES. Leiras Pulpeiro} \\

                        & \small{(Enderezo: Rúa da Orquídea, 45, Lugo)} \\

                        \large{\textbf{Data:}} &
                        \large{14 de Xuño de 2016} \\
		\end{tabular}
	\end{flushright}
        \end{bottompar}

\end{titlepage}
