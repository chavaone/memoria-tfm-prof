\chapter{Fundamentación Teórica}\label{chap:fundamentacion}
Neste capítulo explicaremos a teoría psicolóxica, pedagóxica e sociolóxica sobre a que se asenta a unidade didáctica deseñada para este traballo.

\section{Fundamentación Psicológica}
% Responder a quen vai dirixido
% Aspectos relativos a la psicología personal del los alumnos dirigido a la edad de los alumnos del estudio.
% Desarrollo personal/cognitivo de los estudiantes
% Psicología del aprendizaje. COmo se aprende.
% Autores como Piajet. Brousseau.
Este traballo está orientado a alumnos e alumnas de 1º de ESO. A idade da deles será maioritariamente de entre 11 e 12 anos salvo en determinados casos extraordinarios nos que pode ser inferior, por ter o alumno ou alumna altas capacidades e producirse unha acceleración; ou superior, debido a repetición dun curso ou a procedencia dun sistema educativo extranxeiro no que as competencias acadadas para a súa idade non son suficientes para avanzar de curso. En primeiro lugar faremos unha análise do concepto de adolescencia por estar o noso alumnado dentro dela, a continuación falaremos do desenvolvemento cognitivo dos estudantes desta idade, do desenvolvemento da súa personalidade, explicaremos as teorías relativas a psicoloxía da aprendizaxe e dentro deste último punto dedicaremos especial atención a psicoloxía da aprendizaxe no campo das matemáticas.

\subsection{Pubertade e adolescencia}
Como podemos ver en \citeA{adolescenciacoll} é importante diferenciar entre estos dous termos. Pois mentras a pubertade é un proceso biolóxico que comprenden un ``conxunto de cambios físicos que ao longo da segunda década de vida transforma o corpo infantil nun corpo adulto con capacidade para reproducirse''~(p.~436), a adolescencia é un ``período psicosociolóxico que se prolonga varios anos máis e que se caracteriza pola transición entra a infancia e a adultez''~(p.~436).

Estos autores sinalan que mentras a pubertade é un fenómeno universal que atinxe a toda a especie humana e que causa alteracións físicas sobre todo nos carácteres sexuais primarios (órganos reproductores) e nos secundarios (entre outros, cabelo facial, cambios na voz e ensanchamento dos hombreiros nos varones e crecemento do peito e ensanchamento das cadeiras nas mulleres); mentras que a adolescencia é un fenómeno psicosocial que afecta a unha certa porcentaxe da poboación mundial e cuxo estudo é relativamente recente. A adolescencia como un estadio particular do desenvolvemento non xurde ata finais do século XIX e principios do XX onde a industrialización fai que se lle de máis importancia á formación e os nenos e nenas comecen a pasar máis tempo nas escolas dependendo, durante este tempo, dos pais para a súa subsistencia. Na actulidade a adolescencia vense prolongado notablemente e cada vez máis tarde se adquire o status de adulto. Non obstante noutras sociedades menos desenvolvidas que a nosa, existen ritos que duran varios días ou semanas nos que se pasa de neno a adulto. Neste caso non ten sentido falara de adolescencia no sentido que se lle da no mundo occidental.

No mesmo documento podemos ver que existen varias teorías sobre a adolescencia e mentras algúns autores e autoras fan fincapé en que son os cambios biolóxicos da pubertade os que provocan as transformacións psicolóxica que se viven na adolescencia, outros defenden que é o contexto social o que provoca estos cambios. Tampouco existe consenso nin ningunha teoría que explica que definitivamente o desenvolvemento nesta etapa. Porén, a percepción de que a esta etapa presenta un trance complicado para nenos e nenas está amplamente superada existindo varios estudos de que a maioría de seres humanos non presenta grandes dificultades neste período. En \citeA{adolescenciacoll} vemos que ``a porcentaxe de adolescentes que experimenta algun tipo de desaxuste psicolóxico non supera o 20\%''~(p.~440), sendo este porcentaxe similar ao de problemas na infancia.

\subsection{Desenvolvemento Cognitivo}
En \citeA{shaffer2000psicologia} vemos que na \emph{Teoría do Desenvolvemento Cognitivo} de Jean Piajet existen varias etapas do desenvolvemento cognitivo do ser humano, a etapa \emph{sesoriomotora} (ata os 2 anos), a etapa \emph{preoperatoria} (de 2 a 7 anos), a etapa das \emph{operacións concretas} (de 7 a 11 anos) e a etapa das \emph{operacións formais} (de 11 anos en adiante). Os nenos e nenas de 1º de ESO aos que vai dirixida este traballo teñen entre 11 e 12 anos polo que nos centraremos nas dúas últimas etapas dedicando máis atención á ultima.

 Este libro explica que a etapa das operacións concretas carecterízase por que os nenos e nenas adquiren unha serie de operacións cognitivas que lles permitirán pensar en obxectos e acotecementos que experimentou no pasado. \citeA{piaget1997psicologia} cita como algunha das operacións que se adquiren durante este período: a conservación, a serialización, a clasificación, o número ou o espazo. Estas operacións non están de ningunha forma aisladas e son comunes a todos os individuos dun mesmo nivel mental. Non obstante este tipo de pensamente é limitado pois so se poden aplicar estos esquemas a obxectos reais ou imaxinables sendo incapaces de aplicalas a símbolos abstractos.

Estas limitacións que ten a etapa das operacións concretas son superadas na seguinte etapa, a das operacións formais. En \citeA{shaffer2000psicologia} explicase que esta etapa, que comeza a partir dos 11 anos, fai que os alumnos e alumnas aprendan operacións cognitivas formais e o pensamento deixará de estar vinculado ao observable podendo actuar sobre procesos e feitos hipotéticos. Estos autores sinalan que un operador formal pode traballar con hipóteses e isto é esencial para as matemáticas máis alá da aritmética simple. Nunha ecuación do tipo $2x + 5 = 15$, a variable $x$ non representa algo concreto polo que debe abordarse de forma abstracta.

Nembergantes, existen varias críticas a esta teoría proposta por Piaget como podemos apreciar en \citeA{shaffer2000psicologia}. En primeiro lugar supón que todos os rapaces chegan máis ou menos ao mesmo tempo e da mesma forma a esta etapa das operacións formais. O propio Piaget (1972 en \citeNP{shaffer2000psicologia}, p. 271) nos seus últimos anos de vida mencionou outra posibilidade, todos os estudantes chegan a este nivel pero so nos asuntos que son do seu interese ou que teñan unha importancia vital para eles. Esta hipótese vai en consonancia con estudos feitos posteriormente por outros autores como DeLisi e Staudt (1980 en \citeNP{cognitivocoll}, p.~463) que analizaron as diferencias do grando de alcance das operacións formais dependendo do tipo de preguntas e da formación dos entrevistados.

Por outro lado outra das críticas que recibe esta teoría é a case nula atención que fai as influencias sociais e culturais. O psicólogo ruso Lev Vygotsky formulou outra das teoría máis aceptadas e que si que ten en conta en gran medida as influencias sociais e culturais. Vemos en \citeA{shaffer2000psicologia} que segundo este autor, os nenos e nenas non desenvolven ``o mesmo tipo de mente en todo o mundo''~(p.~274), aprenden a empregar as súas capacidades para a resolución de problemas ``en conformidade coas normas e valores da súa cultura''~(p.~274). O psicólogo ruso creía que as habilidades máis importantes adquiridas polo alumnado proveñen da interación con outras persoas do seu medio, sexan os seus proxenitores, profesores ou profesoras ou membros do seu grupo de iguais.

Este autor defende que nacemos con unhas funcións mentais elementais e que a cultura e a sociedade as transforma en funcións mentais superiores. Para facer isto cada cultura propociona unhas ferramentas de adaptación intelectual que permíten mellorar as súas habilidades. No mesmo libro vemos que outro aspecto importante da teoría de Vygotsky é este concibía a aprendizaxe como cooperativo e identificada que os participantes máis expertos adaptaban o soporte que lle proporcionaban aos novatos en función da súa situación procurando sempre manterse na área de desenvolvemento próximal, que é ``a diferencia entre o que unha persoa pode lograr de forma independente e o que pode lograr cos consellos e estimulos dunha máis diestra''~(p.~278).

\subsection{Desenvolvemento da Personalidade}
Os cambios físicos que os alumnos e alumnas viven na pubertade á súa idade causaran que a súa personalidade se vexa alterada sendo este período fundamental para a definición da súa personaliade futura como podemos ver en \citeA{personalidadcoll}. Algúns dos factores que se verán modificados durante esta época son o autoconcepto, a autoestima, a identidade persoal e o desenvolvemento moral.

Durante os primeiros anos da adolescencia, que é o periodo que viven os estudantes de 1º de ESO aos que lles impartimos clase, producense unha serie de cambios físicos e psiquicos que terán repercusión sobre o seu autoconcepto. Durante estos anos os contidos deste autoconcepto soen estar relacionados con estos cambios que se producen polo que as referencias ao seu aspecto físico serán constantes. Tamén aparecerán frecuentemente as características ou habilidades sociais. En canto á estrutura do autoconcepto, vemos en \citeA{personalidadcoll} que nos primeiros anos da adolescencia surxen unhas ``primeras abstraccións que integran características relacionadas''~(p.~473), estas abstraccións están compartimentadas e grazas a isto non detectarán as incompatibilidades delas evitando desta forma conflictos emocionais.

Relacionado co autoconcepto está a autoestima. Con respecto a isto, o mesmo autor sinala que durante a primeira etapa da adolescencia os niveis de autoestima descenderán de forma xeneralizada debido fundamentalmente a que o ou a adolescente non se sinte satisfeito co seu corpo. O feito de que a sociedade sexa máis esixente co corpo da muller fai que nelas este descenso da autoestima sexa máis acusado. Outras razóns con respecto á menor autoestima está no cambio do colexio ó instituto co que conleva pasar de ser os alumnos de máis idade a ser os de menos, o cambio de profesores e de compañeiros e o aumento da competitividade non so académica senon na práctica de deportes. A todo este coctel hai que engadirlle o inicio das relacións heterosexuais que engadirán mais presión e fará que se sintan máis inseguros.

Como podemos apreciar en Erikson (1968 en \citeNP{personalidadcoll}, p. 478) a construción da identidade persoal é a principal tarefa que deben resolver os adolescentes. Este concepto está relacionado co autoconcepto mais mentras o autoconcepto depende fundamentalmente do desenvolvemento cognitivo, a identidade é un fenómeno psicolóxico. Porén, a construcción da identidade persoal ocupa durante os primeiros anos da adolescencia nun estado de difusión onde o alumando presenta pouco interese en cuestións relacionadas coa identidade.

\subsection{Psicoloxía da aprendizaxe}
% centrarse en una teoria relevante para a propota.
Ao longo do tempo o concepto de aprendizaxe foi definida de moi diferentes formas. Por exemplo, Good~e~Brophy~(1995 en \citeNP{unedpsicoedu}, p.74) definen o aprendizaxe como ``un cambio relativamente permanente na capacidade de execución, adquirida por medio da experiencia''. Outra posible definición é a de Gagné (1976 en \citeNP{unedpsicoedu}, p.74) que a define como ``o profeso que capacita aos organismos a modifica a súa conduta con unha certa rapidez nunha forma máis ou menos permanente''. Independetemente das múltiples definicións que se derón ao longo do tempo, existen uns elementos comúns a todas elas. Según \citeA{unedpsicoedu}, a aprendizaxe consiste nun cambio, é fruto da experiencia e este cambio é relativamente permanente.

Pero, como se produce a aprendizaxe? Sobre esta cuestión teñense dado moitas teorías ao longo do tempo mais podemolas dividir en dúas vertentes diferenciadas. Por un lado temos o enfoque conductual e por outro o enfoque cognitivo.

En \citeA{ertmer1993conductismo} vemos que para as teorías conductistas a aprendizaxe ``lógrase cando se demostra ou se exhibe unha resposa apropiada a continuación da presentación dun estímulo ambiental específico''~(p.~6). Para os conductistas ``toda actividade humana se explica en función de asociccións entre estímulos e respostas''~\cite[p.~82]{unedpsicoedu} o causa que os procesos superiores sexan reducidos a actividade muscular.

Por outro lado no enfoque congnitivo, ``a aprendizaxe equiparase a cambios discretos entre os estados de coñecemento máis que con cambios na probabilidade de resposta''~\cite[p.~9]{ertmer1993conductismo} como ocurria no enfoque conductista. En \citeA{unedpsicoedu} vemos que estas teorías conceden conceden un papel primordial aos procesos de aprendizaxe e preocúpanse na forma en que estos procesos

\subsubsection{Aprendizaxe das matemáticas}



\section{Fundamentación Pedagóxica}
% Responder como imos a ensinar
% Incluir fundamentación metodolóxica e curricular
% Fund. pedagógicos de tipo xeral
% Fund. pedagógicos concretos relacionadas coa temática.
% Tipos de actividades.
% Orientación metodolóxica.
% Justificación curricular (WTF!)


\section{Fundamentación Sociolóxica}
% Respondera a para que imos ensinar.
% Contexto sociocultural
% Sociedad y democracia, etc.
