\subsection{Contribución ao logro das competencias clave}\label{sec:competencias}
Segundo o Decreto 86/2015 as competencias son ``capacidades para aplicar de xeito integrado os contidos propios de cada ensinanza e etapa educativa, co fin de lograr a realización adecuada de actividades e a resolución eficaz de problemas complexos''. Podemos ver en \citeA{aprsuperior} que ``ademais de coñecementos e habilidades, a competencia implica a comprensión do que se fai e saber transferir''~(p.~3).

O decreto citado anteriormente establece no seu Artigo 3 unha serie de competencias denominadas clave e indica que ``para unha adquisición eficaz das competencias [...] deberán deseñarse actividades de aprendizaxe integradas que lle permitan ao alumnado avanzar cara aos resultados de aprendizaxe en máis dunha competencia ao mesmo tempo''. Durante esta sección explicaremos como contribuímos dende esta unidade didáctica ao logro das citadas competencias:

\begin{description}[]
    \item[Comunicación Lingüística (CCL)] Traballaremos esta competencia mellorando o vocabulario do alumnado cos novos conceptos matemáticos aprendidos e tamén a través da súa intervención intervención na clase para expoñer o resultado dos traballos feitos en grupo, para a corrección de problemas no encerado ou para formular e explicar hipóteses sobre como pensan que se debería resolver certos exercicios propostos.
    \item[Comp. matemática e comp. básicas en ciencia e tecnoloxía (CMCCT)] Fomentaremos esta competencia a partir da aprendizaxe do contidos centrais desta unidade onde se aprenderán os principios básicos de xeometría. Introducirase ao alumno no razoamento matemático a partir de certas demostracións gráficas e relacionarase a matemática co mundo real.
    \item[Comp. Dixital (CD)] Promoveremos esta competencia a través do uso de ordenador tanto por parte dos alumnos e alumnas para a realización de actividades como por parte do profesor para facer as exposicións durante a clase. Durante as actividades os alumnos deberán sacar fotografías, enviar correos electrónicos, empregar programas de edición de imaxes para modificar as fotografías e explicar o traballo realizado coa encerado dixital.
    \item[Comp. sociais e cívicas (CSC)] Traballaremos a través do traballo en grupos heteroxéneos onde o alumnado deberá expoñer e defender os resultados do seu traballo.
    \item[Aprender a aprender (CAA)] Fomentaremos esta competencia en menor medida a través da busca de información en internet de pequenas definicións e proporcionando material a partir do cal o alumno pode repasar ou aprender de forma autónoma.
    \item[Conciencia e expresións culturais (CCEC)] Traballaremos a mellor conciencia da cultura propia de Galicia a través do uso do galego como lingua vehicular na clase.
\end{description}

%TODO: mellorar comentarios competencias clave
