% Responder como imos a ensinar
% Incluir fundamentación metodolóxica e curricular
% Fund. pedagógicos de tipo xeral
% Fund. pedagógicos concretos relacionadas coa metodoloxía. papel profesor -alumno.
% Tipos de actividades.
% Orientación metodolóxica.
% Justificación curricular (WTF!)

\section{Fundamentación Pedagóxica}\label{sec:pedago}

A pedagoxía é a ``disciplina que ten como obxecto de estudio a educación coa intención de organizala para cumprir con determindos fins, establecidados a partir dos que é desexable para unha sociedade, é dicir, o tipo de cidadan que se quere formar''~\cite{wiki:pedagogia}. Nesta sección explicaremos os fundamentos pedagóxicos do presente traballo intentando desta forma responder a pregunta de como imos ensinar e porque o imos facer así. Falaremos de aspectos pedagóxicos xerais sobre que tipos de proceso ensino-aprendizaxe pensamos se deberían desenvolver na aula e explicaremos detalles relativos á aprendizaxe das matemáticas e da xeometría.

\subsection{Aspectos pedagóxicos xerais}
Na sección anterior falabamos de Ausubel como a persoa que formulou por primeira vez a teoría da aprendizaxe significativa. En Díaz Barriga~(1989 en \citeNP{arceo1998estrategias}) vemos que este autor entende que a aprendizaxe que se pode dar nunha aula pode definirse en \textbf{dous eixes ou dimensións}. Nun dos eixos contemplamos a forma en que se adquire a información. Esta pode ser por recepción ou por descubrimento. Se a aprendizaxe se produce por recepción producese cando os contidos se representan na súa forma final e se se produce por descubrimento, o contido principal no se da e é o alumnos ou alumna quen debe descubrilo. No outro eixo contemplamos a forma en que o coñecemento se incorpora na estrutura cognitiva do discente. Pode ser significativo, cando o novo coñecemento está directamente relacionado co que xa se posuía ou repetitivo cando se da por asocicións arbitrarias.

É común pensar que a aprendizaxe por recepción tende a ser repetitiva mais isto é un erro. Se se compren as condicións das que falabamos no apartado anterior (motivación e e que o coñecemento este organizado de forma significativa) podese xerar aprendizaxe significativo tanto por descubrimento como por recepción \cite{unedpsicoedu}. Durante a posta en práctica desta unidade didáctica intentaremos conseguir que a aprendizaxe sexa \textbf{significativo e por recepción} posto que este é o tipo de aprendizaxe que defende Ausubel que se debe dar nunha aula aínda que, como tamén explica o autor, tamén se poidan dar en certa proporción aprendizaxes por repetición e por descubrimento.

Outro dos modelos de aprendizaxe que levamos a cabo é o de \textbf{aprendizaxe colaborativo}. Este aprendizaxe no marco do constructivismo, vemos que consiste nun  ``lugar donde los alumnos deben trabajar juntos, ayudándose unos a otros, usando una variedad de instrumentos y recursos informativos que permitan la búsqueda de los objetivos de aprendizaje y actividades para la solución de problemas''~(Wilson, 1995 en \citeNP{calzadilla2002aprendizaje}). Con este tipo de aprendizaxe eliminamos os enfoques repetitivos e de memorización tradicionais e promovemos procesos onde o diálogo entre pares é o protagonista e, ao mesmo tempo, conseguimos que os estudantes se involucren activamente non so na súa propia aprendizaxe senon tamén na dos compañeiros \cite{calzadilla2002aprendizaje, collazos2001aprendizaje}.

Por outro lado o papel do profesor cambia frecuentemente dependendo do tipo de actividades que desenvolvemos nesta unidade didáctica pois, teremos actividades nas que prima a parte expositiva e outras onde destacará o traballo en equipo. Durante as actividades de corte máis tradicional, o profesor terá un papel de \textbf{controlador} no que ``está totalmente ao cargo da clase''~\cite[p.~5]{rubiogarciarolesextranjeros} ou de \textbf{avaliador} durante o cal ``se avaliará o traballo dos estudantes e o seu rendemento, proporcionandose unha retroalimentación importante, polo que os estudantes poden ver o alcance do seu éxito ou fracaso no seu desempeño''~\cite[p.~5]{rubiogarciarolesextranjeros}

Co traballo colaborativo e en equipo vemos que superamos a dependencia do profesor existente en modelo conductistas de aprendizaxe pasando a unha dinámica de grupo moito máis rica \cite{calzadilla2002aprendizaje}. \citeA{collazos2001aprendizaje} explica que a aprendizaxe colaborativa fai que aparezan unha serie de novos roles como poden ser o de \textbf{mediador cognitivo}, axudando ao alumnado a desenvolver o seu pensamento e ser capaces de aprender por eles mesmos; \textbf{deseñador instruccional}, sendo o encargado de crear as actividades que consigan un proceso de ensino-aprendizaxe óptimo; e \textbf{instructor}, levando a cabo as actividades deseñadas previamente dunha forma máis tradicional. Durante esta unidade didáctiva o profesor tomará algún dos papeis anteriores no momento en que se traballe de forma colaborativa.




\subsection{Pedagoxía das matemáticas e da xeometría}
En canto ao traballo en matemáticas en concreto autores como \citeA{alsina1996ensenar} destacan que durante a educación secundaria é necesario ``presentar ao alumnado o mundo da matemática a través da vivencia activa de descubrimentos e reflexións, realizando acticidades e vivindo a aprendizaxe como unha experiencia progresiva, divertida e formativa''~(p.~152). Estos autores sinalan que o traballo que facemos na clase de matemática debería estar composto de amosar ao alumando unha serie de proxectos ou experiencias o mais reais posible e que a través das cales os alumnos e alumnas foran adquirindo as competencias desexadas.

Ademais, estos autores destacan que as metodoloxías que de deben  desenvolver na aula de matemáticas debe facer unha relación frecuente de elementos reais cos conceptos que queremos ensinar, fomentar un progreso dende a intuición ao coñecemento matemático, a comunicación como elemento chave, fomentar actitudes positivas cara ao traballo, así como o traballo traballo grupal, a integración coa realidade cotidiana e o fomento do traballo interdisciplinar. Durante o posta en práctica da nosa unidade didáctica aplicaremos gran parte das liñas metodolóxicas anteriores.

No que respecta ao caso específico da xeometría, en gran medida o Modelo de razonamento xeométrico de Van Hiele. Este modelo, ``explica cómo se produce a evolución do
razoamento xeométrico dos estudantes dividíndoo en cinco niveis consecutivos: a visualización, a análise, a deducción informal, la deducción formal e o rigor''~\cite[p.~81]{vargas2013modelo}. Para os contidos traballados durante esta unidade traballaranse fundamentalmente os primeiros tres niveis. No nivel de visualización, os estudantes serán capaces de recoñecer as figuras xeométricas pero como un todo e sen diferenciar as súas partes. Por outro lado, no segundo nivel, o de análise, o alumnado si que poderá identificar estas partes e analizar as propiedades das figuras pero sen chegar a clasificalas. Por último, no nivel de dedución informal, determinarán figuras polas súas propiedades así como facer interrelacións entre as súas clasificacións (Jaime,~1993~en~\citeNP{vargas2013modelo}).

Os Van Heine tamén propuxeron unha serie de fases para guiar o deseño das actividades de ensino-aprendizaxe de xeometría. Estas fases son: información, orientación dirixida, explicitación, orientación libre e integración. Na primeira fase, a de información, introdúcese aos alumnos e alumnas na materia e avalíanse os coñecementos previos. Na fase de orientación dirixida os estudantes coa axuda do profesor e a través da resolución de problemas adquiren os coñecementos da materia a tratar. Durante a seguinte fase, a de explicitación o alumnado deberá ser capaz e expresar coas súas palabras o resultado do aprendido. Na fase de orientación libre, os alumnos deberán resolver de forma autónoma exercicios empregando os coñecementos aprendidos nas fases anteriores. Na última fase, denominada de integración, realizaranse unha serie de resumos sen introducir contidos novos coa inteción de que os discentes integren todos os novos conceptos dentro do seu esquema mental~(Jaime,~1993~en~\citeNP{vargas2013modelo}). Algunhas das actividades deseñadas neste traballo, se ben non siguen exactamente este esquema proposto, empregan certas ideas introducidas polos autores.
